\documentclass[sans-serif,aspectratio=169]{beamer}
\usepackage[utf8]{inputenc}
\usepackage{array,amsbsy}
\usepackage{tabu}
\usepackage{comment}
\usepackage{soul}
\usepackage{tikz}
\usepackage{tabularray}
\usepackage[absolute,overlay]{textpos}
\usepackage{pifont}
\usepackage{dsfont}
\usepackage[normalem]{ulem}
\usepackage[noend,ruled]{algorithm2e}
\usepackage{tcolorbox}
\usepackage{xparse}
\usepackage{calc}
\usepackage{braket}


\usetikzlibrary{decorations.pathreplacing,decorations.pathmorphing,matrix,positioning}
\usetikzlibrary{shapes,calc,patterns,arrows,shapes,arrows.meta,matrix,math,quotes}


\usetheme{Berlin}
\definecolor{MPIgreen}{HTML}{006C66}
\usecolortheme[named=MPIgreen]{structure}
\useoutertheme{infolines}
\useinnertheme{rectangles}
\usefonttheme{professionalfonts}

\makeatletter
\newcommand\SoulColor{%
  \let\set@color\beamerorig@set@color
  \let\reset@color\beamerorig@reset@color}
\makeatother
\definecolor{darkred}{RGB}{220,20,60}
\setstcolor{darkred}
\newcommand<>{\St}[1]{\alt#2{\SoulColor\st{#1}}{#1}}

%\let\emph\textbf
\let\footnoterule\relax

\setbeamertemplate{section page}
{
    \begin{centering}
    \begin{beamercolorbox}[sep=12pt,center]{part title}
    \usebeamerfont{section title}\insertsection\par
    \end{beamercolorbox}
    \end{centering}
}    
%\setbeamercolor{alerted text}{fg=blue}
\setbeamercolor{section in toc}{fg=black}

\setbeamercolor{passive}{fg=white, use=part title, bg=part title.bg!40}

\setbeamertemplate{headline}{}

\setbeamertemplate{navigation symbols}{} 
\setbeamertemplate{footline}{%
\leavevmode%
\hbox{%

\begin{beamercolorbox}[wd=.30\paperwidth,ht=2.5ex,dp=1.125ex,right]{author
in head/foot}%
\usebeamerfont{title in head/foot}\insertshortauthor\hspace{.3cm}
\end{beamercolorbox}%

\begin{beamercolorbox}[wd=.62\paperwidth,ht=2.5ex,dp=1.125ex,left]{title
in head/foot}
\usebeamerfont{author in head/foot}\hspace{.3cm}\insertshorttitle
\end{beamercolorbox}%

\begin{beamercolorbox}[wd=.08\paperwidth,ht=2.5ex,dp=1.125ex,right]{author
  in head/foot}%
  \usebeamerfont{author in
  head/foot}\insertframenumber/\inserttotalframenumber\hspace{.25cm}
  \end{beamercolorbox}%
}%
\vskip0pt%
}

% \setbeamertemplate{itemize item}{\tiny$\blacksquare$}
% \setitemize{label=\usebeamerfont*{itemize item}%
% \usebeamercolor[fg]{itemize item}
% \usebeamercolor[fg]{itemize subitem}
% \usebeamertemplate{itemize item}}

% \setbeamercolor*{description}{fg=MPIgreen}

\definecolor{darkgreen}{RGB}{0,160,0}
\definecolor{darkred}{RGB}{220,20,60}
\definecolor{darkblue}{RGB}{0,0,160}

\newcommand{\BWT}{\mathsf{BWT}}
\newcommand{\SA}{\mathsf{SA}}
\newcommand{\ISA}{\mathsf{ISA}}
\newcommand{\LCE}{\mathsf{LCE}}
\newcommand{\LCP}{\mathsf{LCP}}
\newcommand{\RMQ}{\mathsf{RMQ}}
\newcommand{\per}{\mathsf{period}}
\renewcommand{\S}{\mathsf{S}}
\newcommand{\C}{\mathsf{C}}
\newcommand{\eps}{\varepsilon}
\newcommand{\RR}{\mathbb{R}}

\newcommand{\ceil}[1]{\lceil #1 \rceil}

\newcommand{\rank}[3]{\mathsf{rank}_{#1,#2}(#3)}
\newcommand{\select}[3]{\mathsf{select}_{#1,#2}(#3)}
\newcommand{\dd}{\mathinner{.\,.}}
\newcommand{\Oh}{\mathcal{O}}
\newcommand{\bigO}{\Oh}
\newcommand{\Ohtilde}{\widetilde{\Oh}}
\newcommand{\where}[1]{{\textcolor{MPIgreen}{[{\footnotesize #1}]}}}
\DeclareMathOperator{\polylog}{polylog}

\makeatletter
\newcommand{\Scale}[2][1]{\scalebox{#1}{$\m@th#2$}}
\makeatother
\newcommand{\sblacktriangleright}{%
  \vcenter{\hbox{\Scale[0.75]{\blacktriangleright}}}%
}
\newcommand{\mycirc}{%
  \textcolor{blue}{%
    \small\raise1.5pt\hbox{\donotcoloroutermaths$\sblacktriangleright$}%
  }%
}
\newcommand{\cmark}{\text{\ding{51}}}

\definecolor{darkgreen}{RGB}{0,180,0}

\newcommand{\hd}{\mathsf{hd}}
\newcommand{\ed}{\mathsf{ed}}
\newcommand{\ded}{\mathsf{dd}}
\newcommand{\td}{\mathsf{td}}
\newcommand{\selfed}{\mathsf{self\text{-}ed}}
\newcommand{\wed}{\ed^w}

\newcommand{\lt}{\mathsf{in}}
\newcommand{\br}{\mathsf{out}}
\newcommand{\dist}{\mathsf{dist}}
\newcommand{\myhfill}{\hskip0pt plus 1filll}

\newcommand{\Exp}{\mathbb{E}}
\newcommand{\tOh}{\Ohtilde}
\newcommand\sed{\mathsf{self}\text{-}\ed}
\newcommand{\len}{\mathsf{len}}
\newcommand{\AG}{\mathsf{AG}}
\newcommand{\BM}{\mathsf{BM}}

\def\twoheadleadsto{\tikz[baseline=(a.base)]{\draw[%
    decorate,decoration={zigzag,segment length=4, amplitude=.9},%
    ] (0,0) -- (.25, 0);%
    \draw[%
    -{Classical TikZ Rightarrow}.{Classical TikZ Rightarrow},%
    ] (.25, 0) -- (.4, 0);%
    \node (a) at (.4/2,-.03) {\phantom{\(\leadsto\)}};%
}}

\newcommand{\onto}{\twoheadleadsto}

\newcommand{\tikzmark}[1]{\tikz[overlay,remember picture] \node (#1) {};}

\makeatletter
\NewDocumentCommand{\DrawBox}{s O{}}{%
    \tikz[overlay,remember picture]{
    \IfBooleanTF{#1}{%
        \coordinate (RightPoint) at ($(left |- right)+(\linewidth-\labelsep-\labelwidth,0.0)$);
    }{%
        \coordinate (RightPoint) at (right.east);
    }%
    \draw[red,#2]
      ($(left)+(-0.2em,0.9em)$) rectangle
      ($(RightPoint)+(0.2em,-0.3em)$);}
}

\NewDocumentCommand{\DrawBoxWide}{s O{}}{%
    \tikz[overlay,remember picture]{
    \IfBooleanTF{#1}{%
        \coordinate (RightPoint) at ($(left |- right)+(\linewidth-\labelsep-\labelwidth,0.0)$);
    }{%
        \coordinate (RightPoint) at (right.east);
    }%
    \draw[red,#2]
      ($(left)+(-\labelwidth,0.9em)$) rectangle
      ($(RightPoint)+(0.2em,-0.3em)$);}
}
\makeatother

\newcommand{\bigpicture}[1]{#1}
%\newcommand{\bigpicture}[1]{}
\newcommand{\thiswork}{\textcolor{darkblue}{[this work]}}



\title[Bounded Edit Distances]{\LARGE Bounded Edit Distance}

\date{Based on joint work with: Alejandro Cassis, Daniel Gibney, Egor Gorbachev, Ce Jin, Sharma Thankachan, and Philip Wellnitz\\\bigskip
\large Algorithms \& Complexity @ Warwick\\  September 22nd, 2025; Coventry, UK}
\author[Tomasz Kociumaka]{\LARGE Tomasz Kociumaka}

\institute{
 \vspace{.4cm}
 \hfill
 \raisebox{-0.5\height}{\includegraphics[width=0.4\textwidth,keepaspectratio]{./pic/mpi-logo-en.pdf}}\hfill
 \raisebox{-0.5\height}{\includegraphics[width=0.4\textwidth,keepaspectratio]{./pic/sic.pdf}}\hfill{\,}
 \vspace{0cm}
}

%\includeonlyframes{current}

\begin{document}

\begin{frame}
\titlepage
\end{frame}

    
\begin{frame}<-4>[label=ed]{\only<6->{Bounded }Edit Distance}

  \begin{block}{Edit distance $\ed(X,Y)$ \hfill Levenshtein distance [Lev65]}
  Minimum number of character insertions, deletions, and substitutions that transform $X$ to $Y$.
  \end{block}
  
  \begin{center}
  \begin{tikzpicture}

        
    \begin{scope}[xscale=0.3,yscale=-1,xshift=-25cm]
    \draw (-1,0) node{$X:$};
    \draw (-1,1) node{$Y:$};
    \foreach \x [count=\i] in {\textcolor{red}{r},e,l,e,\textcolor{red}{v},a,n,t}{
      \draw (\i,0) node {$\mathtt{\vphantom{relevant}\x}$};
    }
    \foreach \x [count=\i] in {e,l,e,\textcolor{red}{p},\textcolor{red}{h},a,n,t}{
        \draw (\i,1) node {$\mathtt{\vphantom{elephant}\x}$};
    }
    \foreach \x/\y in {2/1,3/2,4/3,6/6,7/7,8/8}{
      \draw[thick] (\x,0.2) -- (\y,0.8);
    }
    \foreach \x/\y in {5/4}{
      \draw[thick,red,densely dashed] (\x,0.2) -- (\y,0.8);
    }
  
  
    \draw (14,0.5) node {$\ed(X,Y)=3$};
    \end{scope}
    
    \onslide<2->{
    \begin{scope}[xscale=0.3,yscale=-1]
    \draw (-1,0) node{$X:$};
    \draw (-1,1) node{$Y:$};
    \foreach \x [count=\i] in {b,\textcolor{red}{b},a,b,a,b,\textcolor{red}{b},a,a,b}{
      \draw (\i,0) node {$\mathtt{\vphantom{ab}\x}$};
    }
    \foreach \x [count=\i] in {b,a,b,a,b,\textcolor{red}{a},a,a,b,\textcolor{red}{b}}{
        \draw (\i,1) node {$\mathtt{\vphantom{ab}\x}$};
    }
    \foreach \x/\y in {1/1,3/2,4/3,5/4,6/5,8/7,9/8,10/9}{
      \draw[thick] (\x,0.2) -- (\y,0.9);
    }
    \foreach \x/\y in {7/6}{
      \draw[thick,red,densely dashed] (\x,0.2) -- (\y,0.9);
    }
  
  
    \draw (15,0.5) node {$\ed(X,Y)=3$};
    \end{scope}
    }
  \end{tikzpicture}
  \end{center}
  \pause\pause

  Fundamental measure of string (dis)similarity
  \begin{columns}
  \begin{column}{0.39\textwidth}
    
    \textbf{Applications:}
    \begin{itemize}
        \item Bioinformatics
        \item Natural Language Processing
        \item Version Control Systems
    \end{itemize}
    \vspace{.7cm}
  \end{column}

  \pause
  \begin{column}{0.6\textwidth}
    \vspace{.1cm}

    \textbf{Results:}

    \begin{tblr}{width=.99\linewidth,colspec={X[3,r]X[2,c]X[3.9,l]}}
      %Reference & Time  & Remarks   \\
      \where{Vin68,NW70,Sel74,...} & $\Oh(n^2)$ & $n=\max(|X|,|Y|)$ \\
      \onslide<5->{\where{BI15,BK15}} & \onslide<5->{\textcolor{red}{$n^{2-o(1)}$}} & \onslide<5->{if OVH holds}\\
      \onslide<6->{\where{Ukk85,Mye86}} & \onslide<6->{$\Oh(nk)$} & \onslide<6->{$k=\ed(X,Y)$} \\
      \onslide<7->{\where{LV88,LMS98}} & \onslide<7->{$\Oh(n+k^2)$} & \onslide<8->{\textcolor{red}{OVH-tight}} \\
    \end{tblr}

  \end{column}
  \end{columns}
  
  
  \begin{center}

  \end{center}
  
  \end{frame}

  \begin{frame}{$\Oh(n^2)$-Time Algorithm}

  \begin{center}
    \begin{tikzpicture}[x=0.625cm, y=-0.625cm]
      \draw[white] (-1,-1) -- (11,11);
      \onslide<2,5->{
      \draw[line width=7pt, blue!30,line cap=round](0,0) -- (1,1) -- (2,1) -- (10,9) -- (10,10);
      \begin{scope}[x=0.3cm,y=-1cm,xshift=7.2cm,yshift=-2.75cm]
        \foreach \x [count=\i] in {b,\textcolor{darkred}{b},a,b,a,b,\textcolor{darkred}{b},a,a,b}{
          \draw (\i,0) node {$\mathtt{\vphantom{ab}\x}$};
        }
        \foreach \x [count=\i] in {b,a,b,a,b,\textcolor{darkred}{a},a,a,b,\textcolor{darkred}{b}}{
            \draw (\i,1) node {$\mathtt{\vphantom{ab}\x}$};
        }
        \foreach \x/\y in {1/1,3/2,4/3,5/4,6/5,8/7,9/8,10/9}{
          \draw[thick,darkgreen] (\x,0.2) -- (\y,0.9);
        }
        \foreach \x/\y in {7/6}{
          \draw[thick,darkred,densely dashed] (\x,0.2) -- (\y,0.9);
        }
      \end{scope}
      }
      \onslide<3>{
        \draw[line width=7pt, blue!30,line cap=round](0,0) -- (1,0) -- (9,8) -- (9,9) -- (10,10);
        \begin{scope}[x=0.3cm,y=-1cm,xshift=7.2cm,yshift=-2.75cm]
          \foreach \x [count=\i] in {\textcolor{darkred}{b},b,a,b,a,b,\textcolor{darkred}{b},a,a,b}{
            \draw (\i,0) node {$\mathtt{\vphantom{ab}\x}$};
          }
          \foreach \x [count=\i] in {b,a,b,a,b,\textcolor{darkred}{a},a,a,\textcolor{darkred}{b},b}{
              \draw (\i,1) node {$\mathtt{\vphantom{ab}\x}$};
          }
          \foreach \x/\y in {2/1,3/2,4/3,5/4,6/5,8/7,9/8,10/10}{
            \draw[thick,darkgreen] (\x,0.2) -- (\y,0.9);
          }
          \foreach \x/\y in {7/6}{
            \draw[thick,darkred,densely dashed] (\x,0.2) -- (\y,0.9);
          }
        \end{scope}
        }
      \onslide<1-3>{
      \input{graph.tex}
      }
      \onslide<4->{
        \input{graphdp.tex}
        }

    \end{tikzpicture}
  \end{center}
  
\end{frame}


\againframe<4-6>{ed}

\begin{frame}{$\Oh(nk)$-Time Algorithm}

  \begin{center}
    \begin{tikzpicture}[x=0.625cm, y=-0.625cm]
      \draw[white] (-1,-1) -- (11,11);
      \filldraw<7>[line width=7pt,black!20] (0,0)--(3,0) -- (10,7) -- (10,10)-- (7,10) -- (0,3) -- cycle;
      \onslide<6->{
      \draw[line width=7pt, blue!30,line cap=round](0,0) -- (1,1) -- (2,1) -- (10,9) -- (10,10);
      \begin{scope}[x=0.3cm,y=-1cm,xshift=7.2cm,yshift=-2.75cm]
        \foreach \x [count=\i] in {b,\textcolor{darkred}{b},a,b,a,b,\textcolor{darkred}{b},a,a,b}{
          \draw (\i,0) node {$\mathtt{\vphantom{ab}\x}$};
        }
        \foreach \x [count=\i] in {b,a,b,a,b,\textcolor{darkred}{a},a,a,b,\textcolor{darkred}{b}}{
            \draw (\i,1) node {$\mathtt{\vphantom{ab}\x}$};
        }
        \foreach \x/\y in {1/1,3/2,4/3,5/4,6/5,8/7,9/8,10/9}{
          \draw[thick,darkgreen] (\x,0.2) -- (\y,0.9);
        }
        \foreach \x/\y in {7/6}{
          \draw[thick,darkred,densely dashed] (\x,0.2) -- (\y,0.9);
        }
      \end{scope}
      }

      \input{graphdp_nk.tex}

    \end{tikzpicture}
  \end{center}
  
\end{frame}

\againframe<6-7>{ed}

\begin{frame}<-14,19->{$\Oh(n+k^2)$-Time Algorithm}

  \begin{center}
    \begin{tikzpicture}[x=0.625cm, y=-0.625cm]
      \draw[white] (-1,-1) -- (11,11);
      \onslide<22->{
      \draw[line width=7pt, blue!30,line cap=round](0,0) -- (1,1) -- (2,1) -- (10,9) -- (10,10);
      \begin{scope}[x=0.3cm,y=-1cm,xshift=7.2cm,yshift=-2.75cm]
        \foreach \x [count=\i] in {b,\textcolor{darkred}{b},a,b,a,b,\textcolor{darkred}{b},a,a,b}{
          \draw (\i,0) node {$\mathtt{\vphantom{ab}\x}$};
        }
        \foreach \x [count=\i] in {b,a,b,a,b,\textcolor{darkred}{a},a,a,b,\textcolor{darkred}{b}}{
            \draw (\i,1) node {$\mathtt{\vphantom{ab}\x}$};
        }
        \foreach \x/\y in {1/1,3/2,4/3,5/4,6/5,8/7,9/8,10/9}{
          \draw[thick,darkgreen] (\x,0.2) -- (\y,0.9);
        }
        \foreach \x/\y in {7/6}{
          \draw[thick,darkred,densely dashed] (\x,0.2) -- (\y,0.9);
        }

        \onslide<23>{
          \draw (0,2)[right] node {$\Oh(k^2)$ LCE queries:};
          \draw (0,2.6)[right] node {$\Oh(1)$ time per query};
          \draw (0,3.2)[right] node {$\Oh(n)$ preprocessing time};
        }
      \end{scope}
      }

      \input{graphdp_LV.tex}

    \end{tikzpicture}
  \end{center}
  
\end{frame}

\againframe<7>{ed}


\begin{frame}<1,2>[label=toc]
  \begin{centering}
    \alt<1,2>{
    \begin{beamercolorbox}[sep=10pt,center]{part title}
    \usebeamerfont{section title}Classic algorithms for bounded edit distance\par
    \end{beamercolorbox}
    }{
    \begin{beamercolorbox}[sep=10pt,center]{passive}
    \usebeamerfont{section title}Classic algorithms for bounded edit distance\par
    \end{beamercolorbox}
    }

    \medskip

    \alt<1,3>{
    \begin{beamercolorbox}[sep=10pt,center]{part title}
    \usebeamerfont{section title}Bounded edit distance in various settings\par
    \end{beamercolorbox}
    }{
    \begin{beamercolorbox}[sep=10pt,center]{passive}
    \usebeamerfont{section title}Bounded edit distance in various settings\par
    \end{beamercolorbox}
    }

    \medskip

    \alt<1,4>{
    \begin{beamercolorbox}[sep=10pt,center]{part title}
    \usebeamerfont{section title}Central new tool: Self-edit distance\par
    \end{beamercolorbox}
    }{
    \begin{beamercolorbox}[sep=10pt,center]{passive}
    \usebeamerfont{section title}Central new tool: Self-edit distance\par
    \end{beamercolorbox}
    }

    \medskip

    \alt<1,5>{
    \begin{beamercolorbox}[sep=10pt,center]{part title}
    \usebeamerfont{section title}Algorithm overview\par
    \end{beamercolorbox}
    }{
    \begin{beamercolorbox}[sep=10pt,center]{passive}
    \usebeamerfont{section title}Algorithm overview\par
    \end{beamercolorbox}
    }

    \medskip
    \alt<1,6>{
    \begin{beamercolorbox}[sep=10pt,center]{part title}
    \usebeamerfont{section title}Conclusions and open problems\par
    \end{beamercolorbox}
    }{
    \begin{beamercolorbox}[sep=10pt,center]{passive}
    \usebeamerfont{section title}Conclusions and open problems\par
    \end{beamercolorbox}
    }
    \end{centering}

\end{frame}

\againframe<3>{toc}


\begin{frame}{Compressed String Processing}
    \vspace{-.2cm}
    \hfill \textbf{\small \boldmath Notation: $n = |X| + |Y|$, $k = \ed(X, Y)$, $g = \text{compressed size of } X\cdot Y$}

    \medskip

  \textbf{Problem:} Can we compute (bounded) edit distance faster if the input strings are highly compressible using standard methods? (grammar compression, LZ77, BTW+RLE,\ldots)

\vfill

\bigskip


\begin{center}

\begin{tabular}{rccr}
    \emph{Reference} & \emph{Time} & \emph{Remarks}\\[1ex]
    \hline\\[-1.3ex]
    \where{Vin68,NW70,Sel74,WF74} & $\Oh(n^2)$ &  tight for $g=\Theta(n)$ under SETH \where{BI15,BK15}\\[1ex]
    \onslide<2->{\where{HLLW09,Gaw12,Tis15} & $\Ohtilde(gn)$ & tight for any $g=n^{\Theta(1)}$ under SETH \where{ABBK17}\\[1ex]}
    \onslide<3->{\where{LV88+MSU94} & $\Ohtilde(g + k^2)$ &  \\[1ex]}
    \onslide<4->{\where{G\textbf{K}LS22} & $\Ohtilde(g\sqrt{kn})$ & \\[1ex]}
    \onslide<5->{\textcolor{gray}{\textbf{Open}} & \textcolor{gray}{$\Ohtilde(gk)$} & \\[1ex]}
\end{tabular}
\end{center}
\vfill
\end{frame}

\begin{frame}<-2,5->{Quantum Algorithms}
  \textbf{Problem:}
        Can we compute (bounded) edit distance faster using a quantum computer?

    \medskip
    \pause
    \textbf{Model:}
        Quantum random access oracle implementing the function $i \mapsto S[i]$. \begin{itemize}\item $O_S\ket{i}\ket{0}=\ket{i}\ket{S[i]}$
        \item $O_S$ can be queried with a superposition of indices $i$\end{itemize}

    \pause
    \vfill \bigskip
    \begin{center}
\begin{tabular}{rccr}
    \emph{Reference} & \emph{Time} & \emph{Remarks}\\[1ex]
    \hline\\[-1.3ex]
    \onslide<5->{\where{Vin68,NW70,Sel74,WF74} & $\Oh(n^2)$ & \\[1ex]}
    \onslide<6->{\where{BPS21} &  \textcolor{red}{$n^{1.5-o(1)}$} &  holds for $k=\Theta(n)$ under QSETH\\[1ex]}
    \onslide<7->{\where{LV88+JN23} & $\hat{\Oh}(k\sqrt{n}+k^2)$ & \\[1ex]}
    \onslide<8->{\textbf<10>{\where{JG\textbf{K}T24}} & $\Ohtilde(\sqrt{nk}+k^2)$ &  optimal number $\Ohtilde(\sqrt{nk})$ of oracle queries \\[1ex]}
     \onslide<9->{\textcolor{gray}{\textbf{Open}} & \textcolor{gray}{$\Ohtilde(\sqrt{nk}+k\sqrt{k})$} & \\[1ex]}
\end{tabular}
\end{center}
\vfill
\end{frame}

\begin{frame}{Dynamic Algorithms}
  \textbf{Problem:}
        Maintain strings $X,Y \in \Sigma^{\le n}$ subjects to updates (character edits), and report $\ed(X,Y)$ after each update.
    
      \pause
    \begin{center}
      \begin{tikzpicture}
        
        \begin{scope}[xscale=0.3,yscale=-1]
        \draw (-1,0) node{$X:$};
        \draw (-1,1) node{$Y:$};
    
        \onslide<2>{
        \foreach \x [count=\i] in {b,\textcolor{red}{b},a,b,a,b,\textcolor{red}{b},a,a,b}{
          \draw (\i,0) node {$\mathtt{\vphantom{ab}\x}$};
        }
        \foreach \x [count=\i] in {b,a,b,a,b,\textcolor{red}{a},a,a,b,\textcolor{red}{b}}{
            \draw (\i,1) node {$\mathtt{\vphantom{ab}\x}$};
        }
        \foreach \x/\y in {1/1,3/2,4/3,5/4,6/5,8/7,9/8,10/9}{
          \draw[thick] (\x,0.2) -- (\y,0.9);
        }
        \foreach \x/\y in {7/6}{
          \draw[thick,red,densely dashed] (\x,0.2) -- (\y,0.9);
        }
        \draw (20,0.5) node {$\ed(X,Y)=3$};
        }
        \onslide<3>{
          \foreach \x [count=\i] in {b,b,a,b,a,b,\textcolor{red}{b},a,a,b}{
            \draw (\i,0) node {$\mathtt{\vphantom{ab}\x}$};
          }
          \foreach \x [count=\i] in {b,b,a,b,a,b,\textcolor{red}{a},a,a,b,\textcolor{red}{b}}{
              \draw (\i,1) node {$\mathtt{\vphantom{ab}\x}$};
          }
          \draw (2,1) node[fill=blue!30,inner sep=0pt] {$\mathtt{\vphantom{ab}b}$};

          \foreach \x/\y in {1/1,2/2,3/3,4/4,5/5,6/6,8/8,9/9,10/10}{
            \draw[thick] (\x,0.2) -- (\y,0.9);
          }
          \foreach \x/\y in {7/7}{
            \draw[thick,red,densely dashed] (\x,0.2) -- (\y,0.9);
          }    
          \draw (20,0.5) node {$\ed(X,Y)=2$};
        }
        \onslide<4>{
          \foreach \x [count=\i] in {b,b,a,b,a,b,\textcolor{red}{b},a,b,b}{
            \draw (\i,0) node {$\mathtt{\vphantom{ab}\x}$};
          }
          \draw (9,0) node[fill=blue!30,inner sep=0pt] {$\mathtt{\vphantom{ab}b}$};

          \foreach \x [count=\i] in {b,b,a,b,a,b,\textcolor{red}{a},a,\textcolor{red}{a},b,b}{
              \draw (\i,1) node {$\mathtt{\vphantom{ab}\x}$};
          }
          \foreach \x/\y in {1/1,2/2,3/3,4/4,5/5,6/6,8/8,9/10,10/11}{
            \draw[thick] (\x,0.2) -- (\y,0.9);
          }
          \foreach \x/\y in {7/7}{
            \draw[thick,red,densely dashed] (\x,0.2) -- (\y,0.9);
          }    
          \draw (20,0.5) node {$\ed(X,Y)=2$};
        }
        \onslide<5->{
          \foreach \x [count=\i] in {b,b,\textcolor{red}{a},b,a,b,\textcolor{red}{b},a,b,b}{
            \draw (\i,0) node {$\mathtt{\vphantom{ab}\x}$};
          }
          \foreach \x [count=\i] in {b,b,b,a,b,\textcolor{red}{a},a,\textcolor{red}{a},b,b}{
              \draw (\i,1) node {$\mathtt{\vphantom{ab}\x}$};
          }
          \foreach \x/\y in {1/1,2/2,4/3,5/4,6/5,8/7,9/9,10/10}{
            \draw[thick] (\x,0.2) -- (\y,0.9);
          }
          \foreach \x/\y in {7/6}{
            \draw[thick,red,densely dashed] (\x,0.2) -- (\y,0.9);
          }    
          \draw (20,0.5) node {$\ed(X,Y)=3$};
        }
        \onslide<5>{\draw[ultra thick,blue!30] (2.5, .9) -- (2.5,1.15);}
        \end{scope}
      \end{tikzpicture}
      \end{center}
    \pause\pause\pause\pause

    \vfill
    \begin{center}

\begin{tabular}{rccr}
    \emph{Reference} & \emph{Update Time} & \emph{Remarks}\\[1ex]
    \hline\\[-1.3ex]
    \onslide<6->{\where{Vin68,NW70,Sel74,WF74} & $\Oh(n^2)$ & \\[1ex]}
    \onslide<7->{\where{LV88+MSU94} & $\Ohtilde(k^2)$ &  \\[1ex]}
    \onslide<8->{\where{C\textbf{K}M20} & $\Ohtilde(n)$ &  tight for $k=\Theta(n)$ under OVH\\[1ex]}
    \onslide<9->{\textbf<10>{\where{G\textbf{K}24}} & $\Ohtilde(k)$ &   tight under OVH\\[1ex]}
\end{tabular}
\end{center}
\vfill
\end{frame}

\begin{frame}{Weighted Edit Distance}
    \vfill
    \begin{block}{Weighted Edit Distance $\wed(X,Y)$ \hfill $w \colon (\Sigma \cup \{\varepsilon\}) \times (\Sigma \cup \{\varepsilon\}) \to \RR_{\ge 0}$}
        The minimum cost of transforming $X$ into $Y$ by editing individual characters, where:
        \begin{itemize}
            \item inserting $b$ costs $w(\varepsilon, b)$;
            \item deleting $a$ costs $w(a, \varepsilon)$;
            \item substituting $a$ for $b$ costs $w(a, b)$.
        \end{itemize}
    \end{block}
  \begin{tikzpicture}
      \hspace*{1.3cm}
    \begin{scope}[xscale=0.3,yscale=-1]
    \def\st{10}
    \def\sx{\st-12}
    \def\sy{-0.3}
    \def\sizx{1.3333333}
    \def\sizy{0.4}
    \foreach \x in {0,1,2,3,4}{
        \draw (\sx + \sizx * \x, \sy) -- (\sx + \sizx * \x, \sy + \sizy * 4);
    }
    \foreach \y in {0,1,2,3,4}{
        \draw (\sx, \sy + \sizy * \y) -- (\sx + \sizx * 4, \sy + \sizy * \y);
    }
    \foreach \ch/\r/\c in {w:/1.3/-1.7,\varepsilon/0/1,\mathtt{a}/0/2,\mathtt{b}/0/3,\varepsilon/1/0,0/1/1,1/1/2,4/1/3,\mathtt{a}/2/0,1/2/1,0/2/2,2/2/3,\mathtt{b}/3/0,3/3/1,2/3/2,0/3/3}{
        \draw<1> (\sx + \sizx * \c + \sizx * 0.5, \sy + \sizy * \r + \sizy * 0.5) node {$\mathtt{\vphantom{ab}}\ch$};
    }
    \foreach \ch/\r/\c in {w:/1.3/-1.7,\varepsilon/0/1,\mathtt{a}/0/2,\mathtt{b}/0/3,\varepsilon/1/0,0/1/1,1/1/2,1/1/3,\mathtt{a}/2/0,1/2/1,0/2/2,1/2/3,\mathtt{b}/3/0,1/3/1,1/3/2,0/3/3}{
        \draw<2-> (\sx + \sizx * \c + \sizx * 0.5, \sy + \sizy * \r + \sizy * 0.5) node {$\mathtt{\vphantom{ab}}\ch$};
    }
    

    \draw (\st-1,0) node{$X:$};
    \draw (\st-1,1) node{$Y:$};

    \foreach \x [count=\i] in {b,b,a,b,a,\textcolor{red}{b},b,\textcolor{red}{a},\textcolor{red}{a},b}{
      \draw<1> (\st+\i,0) node {$\mathtt{\vphantom{ab}\x}$};
    }
    \foreach \x [count=\i] in {b,\textcolor{red}{a},b,a,b,a,\textcolor{red}{a},\textcolor{red}{a},b,b}{
        \draw<1> (\st+\i,1) node {$\mathtt{\vphantom{ab}\x}$};
    }
    \foreach \x/\y in {1/1,2/3,3/4,4/5,5/6,7/9,10/10}{
      \draw<1>[thick] (\st+\x,0.2) -- (\st+\y,0.9);
    }
    \foreach \x/\y in {6/7}{
      \draw<1>[thick,red,densely dashed] (\st+\x,0.2) -- (\st+\y,0.9);
    }
    

    \foreach \x [count=\i] in {b,\textcolor{red}{b},a,b,a,b,\textcolor{red}{b},a,a,b}{
      \draw<2-> (\st+\i,0) node {$\mathtt{\vphantom{ab}\x}$};
    }
    \foreach \x [count=\i] in {b,a,b,a,b,\textcolor{red}{a},a,a,b,\textcolor{red}{b}}{
        \draw<2-> (\st+\i,1) node {$\mathtt{\vphantom{ab}\x}$};
    }
    \foreach \x/\y in {1/1,3/2,4/3,5/4,6/5,8/7,9/8,10/9}{
      \draw<2->[thick] (\st+\x,0.2) -- (\st+\y,0.9);
    }
    \foreach \x/\y in {7/6}{
      \draw<2->[thick,red,densely dashed] (\st+\x,0.2) -- (\st+\y,0.9);
    }
    
  
  
    \draw<1> (\st+20,0.5) node {$\wed(X,Y)=6$};
    \draw<2-> (\st+20,0.5) node {$\ed(X,Y)=3$};
    \end{scope}
  \end{tikzpicture}
\end{frame}

\begin{frame}<2-4,6->{Weighted Edit Distance}
    \vspace{-.9cm}
    \hfill \textbf{\small \boldmath Notation: $n = |X| + |Y|$, $k = \wed(X, Y)$}

    \vfill
    \small
\begin{tabular}{rccr}
    \emph{Reference} & \emph{Time} & \emph{Edit weights} & \\[1ex]
    \hline\\[-1.3ex]
    \where{Vin68,...} & $\Oh(n^2)$ & any & tight under OVH\\[1ex]
    \onslide<3->{\where{Ukk85,Mye86} & $\Oh(nk)$ & $\mathbb{R}_{\ge 1}$ & \\[1ex]}
    \onslide<4->{\where{LV88} & $\Oh(n + k^2)$ & $\{1\}$ & tight for $1 \le k \le n$ under OVH\\[1ex]}
    \onslide<6->{\where{DGH\textbf{K}S23} & $\Oh(n + k^5)$ & $\mathbb{R}_{\ge 1}$ & \\[1ex]}
    \onslide<7->{\textbf<11>{\where{C\textbf{K}W23}} & $\tOh(n + \sqrt{nk^3})$ & $\mathbb{R}_{\ge 1}$ & tight for $\sqrt{n} \le k \le n$ under APSPH\\[1ex]}
    \onslide<8->{\textcolor{gray}{\textbf{Open}} & \textcolor{gray}{$\Ohtilde(n+k^{2.5})$} & \textcolor{gray}{$\mathbb{R}_{\ge 1}$} & \\[1ex]}
    \onslide<9->{\textbf<11>{\where{G\textbf{K}24}} & $\tOh(n + Wk^2)$ & $\{1, 2, \ldots, W\}$ & tight for  $1 \le k \le n$ and $W = n^{o(1)}$ under OVH\\[1ex]}
    \onslide<10->{\textcolor{gray}{\textbf{Open}} & \textcolor{gray}{$\Ohtilde(n+k^2 W^{o(1)})$} &\textcolor{gray}{$\{1, 2, \ldots, W\}$}}
\end{tabular}
\end{frame}

\againframe<4>{toc}

\begin{frame}{Self-Edit Distance: Intuition}
    \begin{columns}
        \begin{column}{0.5\textwidth}
        What makes the case $k\ll n$ easier?
        \begin{itemize}
            \item Every optimal alignment $\mathcal{A}$ stays within the $O(k)$ central diagonals.
            \item<2-> Every optimal alignment $\mathcal{A}$ perfectly matches long fragments of $X$ and $Y$.
        \end{itemize}
        \medskip

        \onslide<3->{
            Say $X[i\dd j)=P=Y[i'\dd j')$ for $|i-i'|\le 2k$.
            Can we greedily align these fragments?
            \begin{itemize}
                \item<4-> If $\mathcal{A}$ intersects $(i,j)\leadsto (i',j')$\onslide<5->{, yes!}
                    \begin{itemize}
                        \item<5->[] (For unweighted edit distance.)
                    \end{itemize}
                \item<6-> Otherwise, not necessarily...
                \begin{itemize}
                        \item<7->[] but $P$ has a low-cost \alert{self-alignment}!
                    \end{itemize}
            \end{itemize}
        }
        
        \end{column}
        \begin{column}{0.5\textwidth}
            \smallskip
            \begin{center}
            \begin{tikzpicture}[transform canvas={scale=0.55}, y=-1cm, xshift=-5.3cm, yshift=6cm]
                \bigpicture{
                    
\def\shft{0}
\def\W{12}
\def\H{12}
\def\myi{2}
\def\cnt{6}
\def\gp{2}
\def\kp{1.5}
\def\charwidth{1 / 15}
%\draw[black,thin] (-1,-1) grid (\H + 1,\W + 1);

%0/gray,1/red,2/violet,3/blue,4/magenta,5/orange,6/green,7/yellow


%\foreach \x/\c in {} {
%    \draw<2-4>[fill=\c!40, thick] (\x, {max(\x - \kp, 0)}) rectangle (\x + \gp, {min(\x + \gp + \kp, \H)});
%    \draw<5->[fill=darkblue!40, thick] (\x, {max(\x - \kp, 0)}) rectangle (\x + \gp, {min(\x + \gp + \kp, \H)});
%}

%\foreach \x in {4*\gp} {
%    \draw<2-5>[fill=darkblue!40, thick] (\x, {max(\x - \kp, 0)}) rectangle (\x + \gp, {min(\x + \gp + \kp, \H)});
%    \draw<6->[fill=darkblue!70, thick] (\x, {max(\x - \kp, 0)}) rectangle (\x + \gp, {min(\x + \gp + \kp, \H)});
%}

\draw (\kp,0.1) -- (\kp,-0.1) node[above] {$k$};
\draw (0.1,\kp) -- (-0.1,\kp) node[left] {$k$};
\draw<1>[fill=darkgreen!20] (0, 0) -- (\kp, 0) -- (\H, \W - \kp) -- (\H, \W) -- (\H - \kp, \W) -- (0, \kp) -- (0, 0);
\draw<2->[] (0, 0) -- (\kp, 0) -- (\H, \W - \kp) -- (\H, \W) -- (\H - \kp, \W) -- (0, \kp) -- (0, 0);


\draw[thick] (0, 0) rectangle (\H, \W);

\onslide<1-2>{
    \draw[line width=7pt, darkgreen, cap=round,opacity=0.5] (0,0) -- (0,0.5) -- (3, 3.5) -- (3,4) -- (7,8) -- (8.5,8) -- (10.5,10) -- (10.5,10.5) -- (12,12);
}  
\onslide<2->{
    \draw[line width=4pt, darkblue, cap=round] (3,4) -- (7,8);
}    
\onslide<2->{
\draw[darkblue,ultra thick] (-.2,4) -- (-.2,8) (3, -.2) -- (7, -.2);
\draw[dotted] (-.2, 4) -- (3,4) -- (3, -.2) (-.2, 8) -- (7,8) -- (7, -.2);
}
\onslide<3->{
    \draw (-.2,4) node[left]{{$i'$}};
    \draw (-.2,8) node[left]{{$j'$}};
    \draw (3, -.2) node[above]{$i\vphantom{ij}$};
    \draw (7, -.2) node[above]{$j\vphantom{ij}$};
}

\onslide<4-5>{
    \draw[line width=7pt, darkgreen, cap=round,opacity=0.5] (0,0) -- (0,0.5) -- (5, 5.5) -- (5,6) -- (6,7) -- (6,7.5) -- (8.5,10) -- (9.5,10) -- (10,10.5)-- (10.5,10.5) -- (12,12);
}  
\onslide<5>{
    \draw[line width=5pt, darkred, cap=round,opacity=0.5] (0,0) -- (0,0.5) -- (3,3.5) -- (3,4) -- (7,8) -- (7,8.5) -- (8.5,10) -- (9.5,10) -- (10,10.5)-- (10.5,10.5) -- (12,12);
}  

\onslide<6->{
    \draw[line width=7pt, darkgreen, cap=round,opacity=0.5] (0,0) -- (3.5,3.5) -- (4,3.5) -- (5,4.5) -- (5.5,4.5) -- (6,5) -- (6,5.5) -- (6.5,6) -- (7,6) -- (9,8) -- (9,8.5) -- (10,9.5) -- (10.5,10)-- (10.5,10.5) -- (12,12);
}   
\onslide<7->{
    \draw[line width=5pt, darkred, cap=round,opacity=0.5] (3,4) -- (4.5,4) -- (5,4.5) -- (5.5,4.5) -- (6,5) -- (6,5.5) -- (6.5,6) -- (7,6) -- (7,8);
}   

%\node<4->[darkred] at (\myi * \gp + \gp / 2, \myi * \gp + \gp / 2) {\Large $D_{\myi, {\the\numexpr \myi + 1}}$};


%\draw<6->[fill=orange, thick,opacity=0.4] (4.2 * \gp - 1.0 * \gp * \charwidth, 0) rectangle (4.2 * \gp + 1.0 * \gp * \charwidth, \H);
%\draw<7->[fill=orange, thick,opacity=0.4] (0, 2.5 * \gp - 1.0 * \gp * \charwidth) rectangle (\W, 2.5 * \gp + 1.0 * \gp * \charwidth);

                }
            \end{tikzpicture}
            \end{center}
        \end{column}
    \end{columns}
\end{frame}

 \begin{frame}{Self-Edit Distance}
    \begin{columns}
        \begin{column}{0.5\textwidth}
            \begin{block}{Self-Edit Distance\hfill [C\textbf{K}W23]}
                The self-edit distance $\sed(X)$ of a string $X$ is the distance from $(0, 0)$ to $(|X|, |X|)$ in the unweighted alignment graph of $X$ onto itself with the edges on the main diagonal removed.
            \end{block}
            \onslide<6->{
            \textbf{Intuition:}\hfill (for unweighted edit distance)
            \begin{itemize}
                \item  If $X[i\dd j)=P=Y[i'\dd j')$ with $|i-i'|\le 2k$ and $\sed(P)> 3k$,
            then the two fragments can be aligned greedily.
            \end{itemize}
           
            }
        \end{column}
        \begin{column}{0.5\textwidth}
            \begin{center}
            \begin{tikzpicture}[scale=0.55, y=-1cm, xshift=-5.3cm, yshift=6cm]
                \draw[line width=7pt, white,line cap=round](0, 0) -- (11, 11);
                \draw<3>[line width=7pt, blue!30,line cap=round](0, 0) -- (11, 11);
                %\draw<5>[line width=7pt, blue!30,line cap=round](0, 0) -- (0, 1) -- (2, 3) -- (5, 3) -- (9, 7) -- (9, 10) -- (10, 11) -- (11, 11);
                %\draw<6>[line width=7pt, blue!30,line cap=round](0, 0) -- (1, 0) -- (3, 2) -- (3, 3) -- (5, 3) -- (9, 7) -- (9, 10) -- (10, 11) -- (11, 11);
                %\draw<6-7>[line width=7pt, blue!30,line cap=round](0, 0) -- (1, 0) -- (3, 2) -- (3, 3) -- (5, 3) -- (9, 7) -- (9, 9) -- (10, 9) -- (11, 10) -- (11, 11);
                %\draw<8>[line width=7pt, blue!30,line cap=round](0, 0) -- (1, 0) -- (3, 2) -- (4, 3) -- (5, 3) -- (9, 7) -- (9, 9) -- (10, 9) -- (11, 10) -- (11, 11);
                \draw<5->[line width=7pt, blue!30,line cap=round](0, 0) -- (1, 0) -- (3, 2) -- (4, 3) -- (5, 3) -- (9, 7) -- (9, 8) -- (10, 9) -- (11, 10) -- (11, 11);

                \bigpicture{
                    \input{sed-graph.tex}

                    \draw<2-3>[-latex,very thick,darkgreen] (n0_0) -- (n1_1);
                    \draw<2-3>[-latex,very thick,darkgreen] (n1_1) -- (n2_2);
                    \draw<2-3>[-latex,very thick,darkgreen] (n2_2) -- (n3_3);
                    \draw<2-3>[-latex,very thick,darkgreen] (n3_3) -- (n4_4);
                    \draw<2-3>[-latex,very thick,darkgreen] (n4_4) -- (n5_5);
                    \draw<2-3>[-latex,very thick,darkgreen] (n5_5) -- (n6_6);
                    \draw<2-3>[-latex,very thick,darkgreen] (n6_6) -- (n7_7);
                    \draw<2-3>[-latex,very thick,darkgreen] (n7_7) -- (n8_8);
                    \draw<2-3>[-latex,very thick,darkgreen] (n8_8) -- (n9_9);
                    \draw<2-3>[-latex,very thick,darkgreen] (n9_9) -- (n10_10);
                    \draw<2-3>[-latex,very thick,darkgreen] (n10_10) -- (n11_11);
                }
                %\draw<8->[red, very thick] (-15.5, 10.6) rectangle (-1.7,12.4);
            \end{tikzpicture}
            \end{center}
            %*Picture of $\sed$ alignment graph, flip all parts below main diagonal*
        \end{column}
    \end{columns}
  \end{frame}

\begin{frame}<21-27>{Self-Edit Distance vs Repetitiveness}
    \begin{columns}
        \begin{column}{0.5\textwidth}
            \begin{block}{First Key Observation\hfill [CKW23]}
                If $\sed(X)\le k$, then $X$ can be decomposed into $\le k$ characters and $\le k$ fragments with a \emph{previous occurrence} $\le k$ positions earlier.
            \end{block}
            \bigskip
            \onslide<27->{
            \begin{corollary}           
            If $\sed(X)\le k$, then $\mathsf{LZ77}(X)=\Oh(k)$ and $X$ can be represented by a size-$\Ohtilde(k)$ grammar.
            \end{corollary}
            }
        \end{column}
        \begin{column}{0.5\textwidth}
            \vspace*{-0.4cm}
            \centering
            \begin{tikzpicture}[scale=0.53, y=-1cm]
                \bigpicture{
                    
\def\shft{0}
\def\W{12}
\def\H{12}
\def\myi{2}
\def\cnt{6}
\def\sizeofdot{2pt}
%\draw[black,thin] (-1,-1) grid (\H + 1,\W + 1);

\draw[line width=3pt, white,line cap=round] (0, 0) -- (2, 0) -- (3, 1) -- (3, 2) -- (9, 8) -- (10, 8) -- (12, 10) -- (12, 12);
\draw[line width=3pt, darkgreen!30,line cap=round] (0, 0) -- (2, 0) -- (3, 1) -- (3, 2) -- (9, 8) -- (10, 8) -- (12, 10) -- (12, 12);

\onslide<-14>{

\def\gp{0.12}

\draw[dashed, lightgray] (\gp, \gp) -- (\H - \gp, \W - \gp);

\draw<1>[fill] (0,0) circle  [radius=\sizeofdot];

\draw<2->[blue!50] (0,-\gp) to [out=60,in=120] node[above,midway] {$\mathtt{a}$} (1,-\gp);
\draw<4->[white] (-\gp, 0) to [out=210,in=150] node[left,midway] {\textcolor{blue!50}{$\mathtt{a}$}} (-\gp, 1);
\draw<2>[fill] (1,0) circle  [radius=\sizeofdot];

\draw<3->[red!50] (1,-\gp) to [out=60,in=120] node[above,midway] {$\mathtt{b}$} (2,-\gp);
\draw<5->[white] (-\gp, 1) to [out=210,in=150] node[left,midway] {\textcolor{red!50}{$\mathtt{b}$}} (-\gp, 2);
\draw<3>[fill] (2,0) circle  [radius=\sizeofdot];

\draw<4->[blue!50] (2,-\gp) to [out=60,in=120] node[above,midway] {$\mathtt{a}$} (3,-\gp);
\draw<8->[white] (-\gp, 2) to [out=210,in=150] node[left,midway] {\textcolor{blue!50}{$\mathtt{a}$}} (-\gp, 3);
\draw<4>[fill] (3,1) circle  [radius=\sizeofdot];

\draw<5>[fill] (3,2) circle  [radius=\sizeofdot];

\foreach \x in {4,5,...,9}{
    \draw<8->[white] (\x - 1,-\gp) to [out=60,in=120] node[above,midway] {\textcolor{blue!50}{$\mathtt{a}$}} (\x,-\gp);
    \ifthenelse{\x = 9}{
        \draw<11->[white] (-\gp, \x - 1) to [out=210,in=150] node[left,midway] {\textcolor{blue!50}{$\mathtt{a}$}} (-\gp,\x);
    }{
        \draw<8->[white] (-\gp, \x - 1) to [out=210,in=150] node[left,midway] {\textcolor{blue!50}{$\mathtt{a}$}} (-\gp,\x);
    }
}
\draw<6-8>[dashed, lightgray] (3, \gp) -- (3, 2 - \gp);
\draw<7-8>[dashed, lightgray] (2, \gp) -- (2, 2 - \gp);
\draw<6-8>[dashed, lightgray] (\gp, 2) -- (3 - \gp, 2);
\draw<6-8>[dashed, lightgray] (9, \gp) -- (9, 8 - \gp);
\draw<7-8>[dashed, lightgray] (8, \gp) -- (8, 8 - \gp);
\draw<6-8>[dashed, lightgray] (\gp, 8) -- (9 - \gp, 8);
\draw[white] (3,-\gp) to [out=90,in=90] (9, -\gp);
\draw<6->[violet!50] (3,-\gp) to [out=90,in=90] (9, -\gp);
\draw[white] (-\gp, 2) to [out=180,in=180] (-\gp, 8);
\draw<6-8>[violet!50] (-\gp, 2) to [out=180,in=180] (-\gp, 8);
\draw<7->[violet!50] (2,-\gp) to [out=90,in=90] (8, -\gp);
\draw<6-8>[fill] (9,8) circle  [radius=\sizeofdot];

\draw<9->[cyan!50] (9,-\gp) to [out=60,in=120] node[above,midway] {$\mathtt{c}$} (10,-\gp);
\draw<11->[white] (-\gp, 9) to [out=210,in=150] node[left,midway] {\textcolor{cyan!50}{$\mathtt{c}$}} (-\gp, 10);
\draw<9>[fill] (10,8) circle  [radius=\sizeofdot];

\draw<11->[white] (10,-\gp) to [out=60,in=120] node[above,midway] {\textcolor{blue!50}{$\mathtt{a}$}} (11,-\gp);
\draw<12->[white] (-\gp, 10) to [out=210,in=150] node[left,midway] {\textcolor{blue!50}{$\mathtt{a}$}} (-\gp,11);
\draw<11->[white] (11,-\gp) to [out=60,in=120] node[above,midway] {\textcolor{cyan!50}{$\mathtt{c}$}} (12,-\gp);
\draw<13->[white] (-\gp,11) to [out=210,in=150] node[left,midway] {\textcolor{cyan!50}{$\mathtt{c}$}} (-\gp,12);
\draw<10->[brown!50] (8,-\gp) to [out=60,in=120] (10, -\gp);
\draw<10->[brown!50] (10,-\gp) to [out=60,in=120] (12, -\gp);
\draw<10-11>[fill] (12,10) circle  [radius=\sizeofdot];

\draw<12>[fill] (12,11) circle  [radius=\sizeofdot];
\draw<13>[fill] (12,12) circle  [radius=\sizeofdot];
}

\draw[thick] (0, 0) rectangle (\H, \W);


%\draw[white] (3,-\gp) to [out=60,in=120] (9, -\gp);
%\draw<5> (2,-\gp) to [out=60,in=120] (8, -\gp);

%\foreach \x in {2,3,...,8}
%    \draw<6-> (\x,-\gp) to [out=60,in=120] (\x + 1,-\gp);

%\draw<7->[cyan] (2,-\gp) to [out=90,in=90] (4,-\gp);
%\draw<7->[cyan] (4,-\gp) to [out=90,in=90] (6,-\gp);
%\draw<7>[cyan] (6,-\gp) to [out=90,in=90] (8,-\gp);
%\draw<8>[olive] (6,-\gp) to [out=90,in=90] (9,-\gp);

                    
\def\shft{0}
\def\W{12}
\def\H{12}
\def\myi{2}
\def\cnt{6}
%\draw[black,thin] (-1,-1) grid (\H + 1,\W + 1);


\draw<22->[line width=1.5pt, darkblue] (3, 2) -- (8.75, 7.75);
\draw<25->[line width=1.5pt, darkblue] (2, 2) -- (7.75, 7.75);


\draw<23->[dashed] (3, 2) -- (3, 0);
\draw<23->[dashed] (8.75, 7.75) -- (8.75, 0);

\draw<21->[dashed, lightgray] (0, 0) -- (\H, \W);

\draw<23->[dashed] (3, 2) -- (0, 2);
\draw<23->[dashed] (8.75, 7.75) -- (0, 7.75);

\draw<25->[dashed] (2, 2) -- (2, 0);
\draw<25->[dashed] (7.75, 7.75) -- (7.75, 0);

\def\gp{0.03}
\draw<23->[line width=2pt,darkblue] (3,-0.2) -- (8.75, -0.2);
\onslide<24->{\draw[line width=2pt,darkblue] (-0.2,2) -- (-0.2,7.75);}
\onslide<26->{\draw[line width=2pt,darkblue] (2,-0.4) -- (7.75, -0.4);}

\begin{scope}
    \clip (2,0) rectangle (8.75, -0.5);
\foreach \x in {2,3,...,8}
    \draw<28-> (\x,-\gp) to [out=60,in=120] (\x + 1,-\gp);

\end{scope}
\foreach \x/\y in {0/gray,1/red,2/violet,3/blue,4/magenta,5/orange,6/green,7/yellow}{
    \draw<28>[\y, line width=2pt] (1.5*\x+0.03,-0.4) -- (1.5*\x+1.47,-0.4);
}
\foreach \x/\y in {0/gray,1/red,6/green,7/yellow}{
    \draw<29>[\y, line width=2pt] (1.5*\x+0.03,-0.4) -- (1.5*\x+1.47,-0.4);
}
\draw<29->[cyan,line width=2pt] (3.03,-0.4) -- (4.97,-.4);
\draw<29->[cyan,line width=2pt] (5.03,-.4) -- (6.97,-.4);
\draw<29->[olive,line width=2pt] (7.03,-.4) -- (8.97,-.4);

                }%
            \end{tikzpicture}
        \end{column}
    \end{columns}
  \end{frame}

   \begin{frame}{Self-Edit Distance vs Alignment Intersections}
  \begin{columns}

      \begin{column}{0.5\textwidth}
        \begin{block}{Second Key Observation\hfill [C\textbf{K}W23]}
         If $\sed(X)>2k$, then every two alignments $X \onto Y$ of cost at most $k$ share a common diagonal edge.
        \end{block}
        \pause
        \textbf{Proof:}
        If $\mathcal{A}, \mathcal{B} : X \onto Y$ do not share a diagonal edge, then $\mathcal{A}^{-1}\circ \mathcal{B} : X \onto Y$ is a self-alignment of cost at most $2k$.
        
        \bigskip
        \pause

          \begin{block}{Corollary}
            If $\sed(X[i\dd j))>2k$, then every two alignments $X \onto Y$ of cost at most $k$ \textbf{intersect} within $X[i\dd j)$.
          \end{block}
      \end{column}
      
      
      \begin{column}{0.5\textwidth}
          \begin{tikzpicture}[scale=0.55, y=-1cm]
              \bigpicture{
\def\shft{0}
\def\W{12}
\def\H{12}
\def\myi{2}
\def\cnt{6}
\def\gp{2}
%\draw[black,thin] (-1,-1) grid (\H + 1,\W + 1);




\draw<4-5>[line width=3pt, blue!30!,line cap=round] (0, 0) -- (1, 0) -- (2, 1) -- (2,2) -- (4, 2) -- (5, 3) -- (6,4)-- (6, 5) -- (9, 8) -- (9, 10) -- (11, 12) -- (12, 12);
\draw<6->[line width=3pt, blue!30!,line cap=round] (0, 0) -- (1, 0) -- (2, 1) -- (2,2) -- (2, 4) -- (3, 5) -- (6, 5) -- (9, 8) -- (9, 10) -- (11, 12) -- (12, 12);
\draw[line width=3pt, red!30,line cap=round] (0, 0) -- (0, 1) -- (1, 2) -- (2, 2) -- (8, 8) -- (9, 8) -- (12, 11) -- (12, 12);

\draw[dashed] (\H / 3, 0) -- (\H / 3, \W);
\draw[dashed] (2 * \H / 3, 0) -- (2 * \H / 3, \W);

\draw<5-6>[densely dotted] (2, 0) -- (2,2);
\draw<5-6>[densely dotted] (9, 0) -- (9,8);


\draw[latex-latex] (\H / 3, -0.3) --node[above]{$\sed > 2k$} (2 * \H / 3, -0.3);
\draw<5-6>[latex-latex] (2, -0.15) -- (9, -0.15);

\draw [line width=1.3pt, violet] (0,0) circle [radius=.25];
\draw [line width=1.3pt, violet] (2, 2) circle [radius=.25];
\draw<6> [line width=1.3pt, violet] (5, 5) circle [radius=.25];
\draw<7> [line width=1.3pt, violet, fill=violet] (5, 5) circle [radius=.25];
\draw [line width=1.3pt, violet] (9, 8) circle [radius=.25];
\draw [line width=1.3pt, violet] (\H,\W) circle [radius=.25];


\draw[thick] (0, 0) rectangle (\H, \W);
}
          \end{tikzpicture}
      \end{column}
  \end{columns}
\end{frame}

\againframe<5>{toc}


\begin{frame}<2-15>{Divide-and-Conquer Scheme}
    \begin{columns}
        \begin{column}{0.5\textwidth}

            \smallskip
            \textbf{Simplifying assumption:} We already have a \textcolor{orange!75}{\emph{near-optimal}} alignment of cost $\Oh(k)$.

            \medskip
                \begin{itemize}
                    \item<3->\tikzmark{left}Partition using the \textcolor{orange!75}{\emph{near-optimal}} alignment (e.g., to halve its cost).
                        \item<4-> The concatenation of optimal \textcolor{violet!60}{$X_L\onto Y_L$} and \textcolor{red!50}{$X_R\onto Y_R$} alignments does not need to be an optimal $X\onto Y$ alignment\alt<-6>{.}{\ldots}
                        \item<7-> but it has to be fixed only within a region of small $\sed$ using a local \textcolor{blue}{alignment}.\tikzmark{right}
                        \item<15-> An \textcolor{green}{optimal alignment} can be obtained by following the three alignments.
                    \end{itemize}
                
        \end{column}
        \begin{column}{0.5\textwidth}
            \hfill
            \begin{tikzpicture}[scale=0.5, y=-1cm]
                    \bigpicture{
                        

\def\W{12}
\def\H{12}
\def\cnt{6}
\def\sizeofdot{4pt}
\def\gp{0.0}
%\draw[black,thin] (-1,-1) grid (\H + 1,\W + 1);



\draw<1>[line width=3pt, white, line cap=round] (0, 0) -- (1, 1) -- (1, 1.5) -- (4.5, 5) -- (5.5, 5) -- (6.5, 6) -- (6.5, 8) -- (9, 10.5) -- (10.5, 10.5) -- (\H, \W);


\draw<2->[line width=3pt, orange!30, line cap=round] (0, 0) -- (1, 1) -- (1, 1.5) -- (4.5, 5) -- (5.5, 5) -- (6.5, 6) -- (6.5, 8) -- (9, 10.5) -- (10.5, 10.5) -- (\H, \W);


\node<1>[white] (XL) at (\H / 5, -1) {\Large $X_L$};
\node<1-2> (XL) at (\H / 5, -1) {\textcolor{white}{\Large $X_L$}};
\node<3-> (XL) at (\H / 5, -1) {\Large $X_L$};
\node<3-> (XR) at (4 * \H / 5, -1) {\Large $X_R$};

\node<-2>[white] (YL) at (-1, \W/5) {\Large $Y_L$};
\node<-2>[white] (YR) at (-1, 4*\W/5) {\Large $Y_R$};

\node<3-> (YL) at (-1, \W/5) {\Large $Y_L$};
\node<3-> (YR) at (-1, 4*\W/5) {\Large $Y_R$};



\draw<7->[latex-latex] (\H / 3, -0.3) to node[midway, above] {$\sed = \Theta(k)$} (2 * \H / 3, -0.3);
\draw<7->[latex-latex] (\H / 3, -0.3) -- (\H / 2, -0.3); 
\draw<7->[latex-latex] (2*\H / 3, -0.3) -- (\H / 2, -0.3); 
\draw<7->[dashed, gray] (\H / 3, 0) -- (\H / 3, \W);
\draw<7->[dashed, gray] (2 * \H / 3, 0) -- (2 * \H / 3, \W);

\draw<11->[latex-latex] (\H / 3, -0.3) -- (19* \H / 48, -0.3); 
\draw<11->[latex-latex] (19*\H / 48, -0.3) -- (\H / 2, -0.3);  
\draw<11->[latex-latex] (\H / 2, -0.3) -- (15*\H / 24, -0.3);
\draw<11->[latex-latex] (15*\H / 24, -0.3) -- (2*\H / 3, -0.3);


\draw<11->[dashed, gray] (19*\H / 48, 0) -- (19*\H / 48, \W);
\draw<11->[dashed, gray] (15*\H / 24, 0) -- (15*\H / 24, \W);


\draw<4->[line width=3pt, violet!30, line cap=round] (0, 0) -- (1.5, 0) -- (4, 2.5) -- (4, 3.5) -- (4.5,4) -- (4.5,5) -- (5, 5.5) -- (\H / 2, 5.5);
\draw<5->[line width=3pt, red!30, line cap=round] (\H / 2, 5.5) -- (6, 6.5) -- (7.5, 8) -- (8.5, 8) -- (11, 10.5) -- (11, 11) -- (\H, \W);


\draw<10->[line width=3pt, blue!50,line cap=round] (\H / 3, 4.5) -- (4.5, 5) -- (6.5, 5) -- (7.5, 6) -- (7.5, 8) -- (8, 8.5) -- (2 * \H / 3, 9.5);

\node<11-13>[circle,draw=blue, fill=blue, inner sep=0pt,minimum size=\sizeofdot] (olddot) at (4.5,5) {};
\node<11-13>[circle,draw=blue, fill=blue, inner sep=0pt,minimum size=\sizeofdot] (olddot) at (7.5,8) {};

\node<12-13>[circle,draw=darkgreen, fill=darkgreen, inner sep=0pt,minimum size=\sizeofdot] (olddot) at (5.5,5.5) {};
\node<12-13>[circle,draw=darkgreen, fill=darkgreen, inner sep=0pt,minimum size=\sizeofdot] (olddot) at (7,7.5) {};

\node<14->[circle,draw=darkgreen, fill=darkgreen, inner sep=0pt,minimum size=\sizeofdot] (olddot) at (4.5,5) {};
\node<14->[circle,draw=darkgreen, fill=darkgreen, inner sep=0pt,minimum size=\sizeofdot] (olddot) at (7.5,8) {};

\draw<13>[line width=1.5pt, darkgreen!70, line cap=round] (0, 0) -- (1.5, 0) -- (4, 2.5) -- (4, 3.5) -- (4.5, 4) -- (4.5,5)-- (5,5.5) -- (5.5, 5.5);

\draw<13>[line width=1.5pt, darkgreen!70, line cap=round] (7, 7.5) -- (7.5, 8) -- (8.5, 8) -- (11, 10.5) -- (11, 11) -- (\H, \W);


\draw<14>[line width=1.5pt, darkgreen!70, line cap=round] (0, 0) -- (1.5, 0) -- (4, 2.5) -- (4, 3.5) -- (4.5, 4) -- (4.5,5);

\draw<14>[line width=1.5pt, darkgreen!70, line cap=round] (7.5, 8) -- (8.5, 8) -- (11, 10.5) -- (11, 11) -- (\H, \W);

\draw<15->[line width=1.5pt, darkgreen!70, line cap=round](0, 0) -- (1.5, 0) -- (4, 2.5) -- (4, 3.5) -- (4.5, 4) -- (4.5,5) -- (6.5, 5) -- (7.5, 6) -- (7.5, 8) -- (8.5, 8) -- (11, 10.5) -- (11, 11) -- (\H, \W);

\draw<3->[dashed, gray] (0, 5.5) -- (\H, 5.5);

\draw<4>[very thick, violet, fill=violet, fill opacity=0.2] (0, 0) rectangle (\H / 2, 5.5);
\draw<5>[very thick, red, fill=red, fill opacity=0.2] (\H / 2, 5.5) rectangle (\H, \W);

\node<3->[circle,draw=orange, fill=orange, inner sep=0pt,minimum size=\sizeofdot] (olddot) at (\H / 2, 5.5) {};

\draw[thick] (0, 0) rectangle (\H, \W);
\draw<3-> (\H / 2, 0) -- (\H / 2, \W);
\draw<9>[blue, thick] (\H / 3, 4.5) rectangle (2 * \H / 3, 9.5);
\draw<10>[very thick, blue, fill= blue, fill opacity=0.2] (\H / 3, 4.5) rectangle (2 * \H / 3, 9.5);



                    }

            \end{tikzpicture}
            \hfill
            %*Picture of some alignment $\mathcal{A}$, optimal alignments for parts*
        \end{column}
    \end{columns}
  \end{frame}


  \begin{frame}{Divide-and-Conquer Scheme: Consequences}
    \textbf{Main Takeaway}
   \begin{enumerate}
    \item Computing $\ed_{\le k}(X,Y)$ reduces to the case of $\sed(X), \sed(Y)=\Oh(k)$.
    \begin{itemize}
        \item<2-> In the weighted and dynamic setting, one even can avoid extra logarithmic factors.
    \end{itemize}
    \item<3-> Due to $\mathsf{LZ77}(S)=\Oh(\sed(S))$, we can use tools from the fully compressed setting.
\end{enumerate}
\pause\pause\pause\bigskip

\textbf{Notes}
\begin{itemize}
    \item<4-> Quantum setting:
    \begin{itemize}
        \item<4-> The divide-and-conquer scheme is more involved to avoid near-optimal alignment.
        \item<5-> \emph{Oracle identification} can ``read'' a string compressible to size $g$ using $\Ohtilde(\sqrt{ng})$ queries.
        \item<5-> For LZ77, also $\Ohtilde(\sqrt{ng})$ time is enough.
        \item<6-> Once we read $X$ and $Y$, we can apply the classic $\Oh(g+k^2)$-time algorithm.
    \end{itemize}
    \item<7->Dynamic and weighted settings:
    \begin{itemize}
        \item<7-> The divide-and-conquer scheme can be implemented without any logarithmic overheads.
        \item<8-> Tools needed for processing compressed strings: 
        \begin{enumerate}
            \item Distance matrices.
            \item Min-plus multiplication of Monge matrices.
            \item Weight-balanced grammars.
        \end{enumerate}
    \end{itemize}
   \end{itemize}
  \end{frame}


\begin{frame}<-5,9-11>{Distance Matrix}
  \vspace{-.3cm}
  \begin{columns}%
    \begin{column}{.5\textwidth}%
  \begin{center}
    \begin{tikzpicture}[x=0.625cm, y=-0.625cm]
      \draw[white] (-1,-1) -- (11,11);

      \draw<3,9>[line width=7pt, blue!30,line cap=round](0,0) -- (1,1) -- (2,1) -- (10,9) -- (10,10);
      \draw<4,9>[line width=7pt, teal!30,line cap=round](0,2) -- (1,3) -- (2,3) -- (5,6) -- (8,6) -- (9,7) -- (10,7);
      \draw<10>[line width=7pt, orange!30,line cap=round](0,0) -- (1,1) -- (2,1) -- (7,6) -- (8,6) -- (9,7) -- (10,7);
      \draw<10>[line width=7pt, red!30,line cap=round](0,2) -- (1,3) -- (2,3) -- (5,6) -- (7,6) -- (10,9) -- (10,10);
      \draw<11>[line width=7pt, orange!30,line cap=round](0,0) -- (1,1) -- (2,1) -- (6,5) -- (7,5) -- (9,7) -- (10,7);
      \draw<11>[line width=7pt, red!30,line cap=round](0,2) -- (1,3) -- (2,3) -- (5,6) -- (7,6) -- (10,9) -- (10,10);
      \draw<6-7>[line width=7pt, blue!30,line cap=round](0,2) -- (1,3) -- (2,3) -- (1,2) -- (1,1) -- (0,0) -- (1,0) -- (6,5) -- (7,5)-- (9,7) -- (10,7);
      \draw<7>[line width=7pt, blue!60,line cap=round](2,3) -- (4,3);


      \bigpicture{\input{graph.tex}}

      \foreach \y[evaluate={\lb=int(10-\y);}] in {1,...,10}{
        \draw (n0_\y) node[above=-0.01] {\scriptsize $\lt_{\lb}$};
      }
      \foreach \x[evaluate={\lb=int(10+\x);}] in {0,...,10}{
        \draw (n\x_0) node[above=-0.01] {\scriptsize $\lt_{\lb}$};
      }

      \foreach \y[evaluate={\lb=int(20-\y);}] in {0,...,9}{
        \draw (n10_\y) node[below=-0.01] {\scriptsize $\br_{\lb}$};
      }
      \foreach \x in {0,...,10}{
        \draw (n\x_10) node[below=-0.01] {\scriptsize $\br_{\x}$};
      }

    

    \end{tikzpicture}
  \end{center}
\end{column}%
  \begin{column}{.5\textwidth}%
    \begin{center}   
      $D_{X,Y}[i,j] = \dist(\lt_i,\br_j)$
      \pause
      \begin{tikzpicture}[xscale=0.3, yscale=-0.3]

        \fill<3,8->[blue!30] (9.5,9.5) rectangle (10.5, 10.5);
        \fill<4,6->[teal!30] (12.5,7.5) rectangle (13.5, 8.5);

        \fill<10->[orange!30] (12.5,9.5) rectangle (13.5, 10.5);
        \fill<10->[red!30] (9.5,7.5) rectangle (10.5, 8.5);

        \bigpicture{\draw (0, 0) node {\tiny $0$};
\draw (1, 0) node {\tiny $1$};
\draw (2, 0) node {\tiny $2$};
\draw (3, 0) node {\tiny $3$};
\draw (4, 0) node {\tiny $4$};
\draw (5, 0) node {\tiny $5$};
\draw (6, 0) node {\tiny $6$};
\draw (7, 0) node {\tiny $7$};
\draw (8, 0) node {\tiny $8$};
\draw (9, 0) node {\tiny $9$};
\draw (10, 0) node {\tiny $10$};
\draw (11, 0) node {\tiny $\infty$};
\draw (12, 0) node {\tiny $\infty$};
\draw (13, 0) node {\tiny $\infty$};
\draw (14, 0) node {\tiny $\infty$};
\draw (15, 0) node {\tiny $\infty$};
\draw (16, 0) node {\tiny $\infty$};
\draw (17, 0) node {\tiny $\infty$};
\draw (18, 0) node {\tiny $\infty$};
\draw (19, 0) node {\tiny $\infty$};
\draw (20, 0) node {\tiny $\infty$};
\draw (0, 1) node {\tiny $1$};
\draw (1, 1) node {\tiny $0$};
\draw (2, 1) node {\tiny $1$};
\draw (3, 1) node {\tiny $2$};
\draw (4, 1) node {\tiny $3$};
\draw (5, 1) node {\tiny $4$};
\draw (6, 1) node {\tiny $5$};
\draw (7, 1) node {\tiny $6$};
\draw (8, 1) node {\tiny $7$};
\draw (9, 1) node {\tiny $8$};
\draw (10, 1) node {\tiny $9$};
\draw (11, 1) node {\tiny $10$};
\draw (12, 1) node {\tiny $\infty$};
\draw<30-> (12, 1) node {\tiny $11$};
\draw (13, 1) node {\tiny $\infty$};
\draw<30-> (13, 1) node {\tiny $12$};
\draw (14, 1) node {\tiny $\infty$};
\draw<30-> (14, 1) node {\tiny $13$};
\draw (15, 1) node {\tiny $\infty$};
\draw<30-> (15, 1) node {\tiny $14$};
\draw (16, 1) node {\tiny $\infty$};
\draw<30-> (16, 1) node {\tiny $15$};
\draw (17, 1) node {\tiny $\infty$};
\draw<30-> (17, 1) node {\tiny $16$};
\draw (18, 1) node {\tiny $\infty$};
\draw<30-> (18, 1) node {\tiny $17$};
\draw (19, 1) node {\tiny $\infty$};
\draw<30-> (19, 1) node {\tiny $18$};
\draw (20, 1) node {\tiny $\infty$};
\draw<30-> (20, 1) node {\tiny $19$};
\draw (0, 2) node {\tiny $2$};
\draw (1, 2) node {\tiny $1$};
\draw (2, 2) node {\tiny $0$};
\draw (3, 2) node {\tiny $1$};
\draw (4, 2) node {\tiny $2$};
\draw (5, 2) node {\tiny $3$};
\draw (6, 2) node {\tiny $4$};
\draw (7, 2) node {\tiny $5$};
\draw (8, 2) node {\tiny $6$};
\draw (9, 2) node {\tiny $7$};
\draw (10, 2) node {\tiny $8$};
\draw (11, 2) node {\tiny $9$};
\draw (12, 2) node {\tiny $10$};
\draw (13, 2) node {\tiny $\infty$};
\draw<30-> (13, 2) node {\tiny $11$};
\draw (14, 2) node {\tiny $\infty$};
\draw<30-> (14, 2) node {\tiny $12$};
\draw (15, 2) node {\tiny $\infty$};
\draw<30-> (15, 2) node {\tiny $13$};
\draw (16, 2) node {\tiny $\infty$};
\draw<30-> (16, 2) node {\tiny $14$};
\draw (17, 2) node {\tiny $\infty$};
\draw<30-> (17, 2) node {\tiny $15$};
\draw (18, 2) node {\tiny $\infty$};
\draw<30-> (18, 2) node {\tiny $16$};
\draw (19, 2) node {\tiny $\infty$};
\draw<30-> (19, 2) node {\tiny $17$};
\draw (20, 2) node {\tiny $\infty$};
\draw<30-> (20, 2) node {\tiny $18$};
\draw (0, 3) node {\tiny $3$};
\draw (1, 3) node {\tiny $2$};
\draw (2, 3) node {\tiny $1$};
\draw (3, 3) node {\tiny $2$};
\draw (4, 3) node {\tiny $2$};
\draw (5, 3) node {\tiny $3$};
\draw (6, 3) node {\tiny $3$};
\draw (7, 3) node {\tiny $4$};
\draw (8, 3) node {\tiny $5$};
\draw (9, 3) node {\tiny $6$};
\draw (10, 3) node {\tiny $7$};
\draw (11, 3) node {\tiny $8$};
\draw (12, 3) node {\tiny $9$};
\draw (13, 3) node {\tiny $10$};
\draw (14, 3) node {\tiny $\infty$};
\draw<30-> (14, 3) node {\tiny $11$};
\draw (15, 3) node {\tiny $\infty$};
\draw<30-> (15, 3) node {\tiny $12$};
\draw (16, 3) node {\tiny $\infty$};
\draw<30-> (16, 3) node {\tiny $13$};
\draw (17, 3) node {\tiny $\infty$};
\draw<30-> (17, 3) node {\tiny $14$};
\draw (18, 3) node {\tiny $\infty$};
\draw<30-> (18, 3) node {\tiny $15$};
\draw (19, 3) node {\tiny $\infty$};
\draw<30-> (19, 3) node {\tiny $16$};
\draw (20, 3) node {\tiny $\infty$};
\draw<30-> (20, 3) node {\tiny $17$};
\draw (0, 4) node {\tiny $4$};
\draw (1, 4) node {\tiny $3$};
\draw (2, 4) node {\tiny $2$};
\draw (3, 4) node {\tiny $3$};
\draw (4, 4) node {\tiny $3$};
\draw (5, 4) node {\tiny $3$};
\draw (6, 4) node {\tiny $3$};
\draw (7, 4) node {\tiny $3$};
\draw (8, 4) node {\tiny $4$};
\draw (9, 4) node {\tiny $5$};
\draw (10, 4) node {\tiny $6$};
\draw (11, 4) node {\tiny $7$};
\draw (12, 4) node {\tiny $8$};
\draw (13, 4) node {\tiny $9$};
\draw (14, 4) node {\tiny $10$};
\draw (15, 4) node {\tiny $\infty$};
\draw<30-> (15, 4) node {\tiny $11$};
\draw (16, 4) node {\tiny $\infty$};
\draw<30-> (16, 4) node {\tiny $12$};
\draw (17, 4) node {\tiny $\infty$};
\draw<30-> (17, 4) node {\tiny $13$};
\draw (18, 4) node {\tiny $\infty$};
\draw<30-> (18, 4) node {\tiny $14$};
\draw (19, 4) node {\tiny $\infty$};
\draw<30-> (19, 4) node {\tiny $15$};
\draw (20, 4) node {\tiny $\infty$};
\draw<30-> (20, 4) node {\tiny $16$};
\draw (0, 5) node {\tiny $5$};
\draw (1, 5) node {\tiny $4$};
\draw (2, 5) node {\tiny $3$};
\draw (3, 5) node {\tiny $4$};
\draw (4, 5) node {\tiny $3$};
\draw (5, 5) node {\tiny $3$};
\draw (6, 5) node {\tiny $3$};
\draw (7, 5) node {\tiny $3$};
\draw (8, 5) node {\tiny $4$};
\draw (9, 5) node {\tiny $5$};
\draw (10, 5) node {\tiny $6$};
\draw (11, 5) node {\tiny $6$};
\draw (12, 5) node {\tiny $7$};
\draw (13, 5) node {\tiny $8$};
\draw (14, 5) node {\tiny $9$};
\draw (15, 5) node {\tiny $10$};
\draw (16, 5) node {\tiny $\infty$};
\draw<30-> (16, 5) node {\tiny $11$};
\draw (17, 5) node {\tiny $\infty$};
\draw<30-> (17, 5) node {\tiny $12$};
\draw (18, 5) node {\tiny $\infty$};
\draw<30-> (18, 5) node {\tiny $13$};
\draw (19, 5) node {\tiny $\infty$};
\draw<30-> (19, 5) node {\tiny $14$};
\draw (20, 5) node {\tiny $\infty$};
\draw<30-> (20, 5) node {\tiny $15$};
\draw (0, 6) node {\tiny $6$};
\draw (1, 6) node {\tiny $5$};
\draw (2, 6) node {\tiny $4$};
\draw (3, 6) node {\tiny $4$};
\draw (4, 6) node {\tiny $3$};
\draw (5, 6) node {\tiny $3$};
\draw (6, 6) node {\tiny $3$};
\draw (7, 6) node {\tiny $2$};
\draw (8, 6) node {\tiny $3$};
\draw (9, 6) node {\tiny $4$};
\draw (10, 6) node {\tiny $5$};
\draw (11, 6) node {\tiny $5$};
\draw (12, 6) node {\tiny $6$};
\draw (13, 6) node {\tiny $7$};
\draw (14, 6) node {\tiny $8$};
\draw (15, 6) node {\tiny $9$};
\draw (16, 6) node {\tiny $10$};
\draw (17, 6) node {\tiny $\infty$};
\draw<30-> (17, 6) node {\tiny $11$};
\draw (18, 6) node {\tiny $\infty$};
\draw<30-> (18, 6) node {\tiny $12$};
\draw (19, 6) node {\tiny $\infty$};
\draw<30-> (19, 6) node {\tiny $13$};
\draw (20, 6) node {\tiny $\infty$};
\draw<30-> (20, 6) node {\tiny $14$};
\draw (0, 7) node {\tiny $7$};
\draw (1, 7) node {\tiny $6$};
\draw (2, 7) node {\tiny $5$};
\draw (3, 7) node {\tiny $5$};
\draw (4, 7) node {\tiny $4$};
\draw (5, 7) node {\tiny $4$};
\draw (6, 7) node {\tiny $3$};
\draw (7, 7) node {\tiny $2$};
\draw (8, 7) node {\tiny $3$};
\draw (9, 7) node {\tiny $4$};
\draw (10, 7) node {\tiny $5$};
\draw (11, 7) node {\tiny $4$};
\draw (12, 7) node {\tiny $5$};
\draw (13, 7) node {\tiny $6$};
\draw (14, 7) node {\tiny $7$};
\draw (15, 7) node {\tiny $8$};
\draw (16, 7) node {\tiny $9$};
\draw (17, 7) node {\tiny $10$};
\draw (18, 7) node {\tiny $\infty$};
\draw<30-> (18, 7) node {\tiny $11$};
\draw (19, 7) node {\tiny $\infty$};
\draw<30-> (19, 7) node {\tiny $12$};
\draw (20, 7) node {\tiny $\infty$};
\draw<30-> (20, 7) node {\tiny $13$};
\draw (0, 8) node {\tiny $8$};
\draw (1, 8) node {\tiny $7$};
\draw (2, 8) node {\tiny $6$};
\draw (3, 8) node {\tiny $5$};
\draw (4, 8) node {\tiny $4$};
\draw (5, 8) node {\tiny $4$};
\draw (6, 8) node {\tiny $3$};
\draw (7, 8) node {\tiny $2$};
\draw (8, 8) node {\tiny $3$};
\draw (9, 8) node {\tiny $4$};
\draw (10, 8) node {\tiny $4$};
\draw (11, 8) node {\tiny $3$};
\draw (12, 8) node {\tiny $4$};
\draw (13, 8) node {\tiny $5$};
\draw (14, 8) node {\tiny $6$};
\draw (15, 8) node {\tiny $7$};
\draw (16, 8) node {\tiny $8$};
\draw (17, 8) node {\tiny $9$};
\draw (18, 8) node {\tiny $10$};
\draw (19, 8) node {\tiny $\infty$};
\draw<30-> (19, 8) node {\tiny $11$};
\draw (20, 8) node {\tiny $\infty$};
\draw<30-> (20, 8) node {\tiny $12$};
\draw (0, 9) node {\tiny $9$};
\draw (1, 9) node {\tiny $8$};
\draw (2, 9) node {\tiny $7$};
\draw (3, 9) node {\tiny $6$};
\draw (4, 9) node {\tiny $5$};
\draw (5, 9) node {\tiny $5$};
\draw (6, 9) node {\tiny $4$};
\draw (7, 9) node {\tiny $3$};
\draw (8, 9) node {\tiny $4$};
\draw (9, 9) node {\tiny $5$};
\draw (10, 9) node {\tiny $4$};
\draw (11, 9) node {\tiny $3$};
\draw (12, 9) node {\tiny $4$};
\draw (13, 9) node {\tiny $4$};
\draw (14, 9) node {\tiny $5$};
\draw (15, 9) node {\tiny $6$};
\draw (16, 9) node {\tiny $7$};
\draw (17, 9) node {\tiny $8$};
\draw (18, 9) node {\tiny $9$};
\draw (19, 9) node {\tiny $10$};
\draw (20, 9) node {\tiny $\infty$};
\draw<30-> (20, 9) node {\tiny $11$};
\draw (0, 10) node {\tiny $10$};
\draw (1, 10) node {\tiny $9$};
\draw (2, 10) node {\tiny $8$};
\draw (3, 10) node {\tiny $7$};
\draw (4, 10) node {\tiny $6$};
\draw (5, 10) node {\tiny $5$};
\draw (6, 10) node {\tiny $4$};
\draw (7, 10) node {\tiny $3$};
\draw (8, 10) node {\tiny $4$};
\draw (9, 10) node {\tiny $4$};
\draw (10, 10) node {\tiny $3$};
\draw (11, 10) node {\tiny $2$};
\draw (12, 10) node {\tiny $3$};
\draw (13, 10) node {\tiny $3$};
\draw (14, 10) node {\tiny $4$};
\draw (15, 10) node {\tiny $5$};
\draw (16, 10) node {\tiny $6$};
\draw (17, 10) node {\tiny $7$};
\draw (18, 10) node {\tiny $8$};
\draw (19, 10) node {\tiny $9$};
\draw (20, 10) node {\tiny $10$};
\draw (0, 11) node {\tiny $\infty$};
\draw<30-> (0, 11) node {\tiny $11$};
\draw (1, 11) node {\tiny $10$};
\draw (2, 11) node {\tiny $9$};
\draw (3, 11) node {\tiny $8$};
\draw (4, 11) node {\tiny $7$};
\draw (5, 11) node {\tiny $6$};
\draw (6, 11) node {\tiny $5$};
\draw (7, 11) node {\tiny $4$};
\draw (8, 11) node {\tiny $4$};
\draw (9, 11) node {\tiny $3$};
\draw (10, 11) node {\tiny $2$};
\draw (11, 11) node {\tiny $1$};
\draw (12, 11) node {\tiny $2$};
\draw (13, 11) node {\tiny $2$};
\draw (14, 11) node {\tiny $3$};
\draw (15, 11) node {\tiny $4$};
\draw (16, 11) node {\tiny $5$};
\draw (17, 11) node {\tiny $6$};
\draw (18, 11) node {\tiny $7$};
\draw (19, 11) node {\tiny $8$};
\draw (20, 11) node {\tiny $9$};
\draw (0, 12) node {\tiny $\infty$};
\draw<30-> (0, 12) node {\tiny $12$};
\draw (1, 12) node {\tiny $\infty$};
\draw<30-> (1, 12) node {\tiny $11$};
\draw (2, 12) node {\tiny $10$};
\draw (3, 12) node {\tiny $9$};
\draw (4, 12) node {\tiny $8$};
\draw (5, 12) node {\tiny $7$};
\draw (6, 12) node {\tiny $6$};
\draw (7, 12) node {\tiny $5$};
\draw (8, 12) node {\tiny $5$};
\draw (9, 12) node {\tiny $4$};
\draw (10, 12) node {\tiny $3$};
\draw (11, 12) node {\tiny $2$};
\draw (12, 12) node {\tiny $3$};
\draw (13, 12) node {\tiny $3$};
\draw (14, 12) node {\tiny $4$};
\draw (15, 12) node {\tiny $3$};
\draw (16, 12) node {\tiny $4$};
\draw (17, 12) node {\tiny $5$};
\draw (18, 12) node {\tiny $6$};
\draw (19, 12) node {\tiny $7$};
\draw (20, 12) node {\tiny $8$};
\draw (0, 13) node {\tiny $\infty$};
\draw<30-> (0, 13) node {\tiny $13$};
\draw (1, 13) node {\tiny $\infty$};
\draw<30-> (1, 13) node {\tiny $12$};
\draw (2, 13) node {\tiny $\infty$};
\draw<30-> (2, 13) node {\tiny $11$};
\draw (3, 13) node {\tiny $10$};
\draw (4, 13) node {\tiny $9$};
\draw (5, 13) node {\tiny $8$};
\draw (6, 13) node {\tiny $7$};
\draw (7, 13) node {\tiny $6$};
\draw (8, 13) node {\tiny $5$};
\draw (9, 13) node {\tiny $4$};
\draw (10, 13) node {\tiny $3$};
\draw (11, 13) node {\tiny $2$};
\draw (12, 13) node {\tiny $2$};
\draw (13, 13) node {\tiny $2$};
\draw (14, 13) node {\tiny $3$};
\draw (15, 13) node {\tiny $2$};
\draw (16, 13) node {\tiny $3$};
\draw (17, 13) node {\tiny $4$};
\draw (18, 13) node {\tiny $5$};
\draw (19, 13) node {\tiny $6$};
\draw (20, 13) node {\tiny $7$};
\draw (0, 14) node {\tiny $\infty$};
\draw<30-> (0, 14) node {\tiny $14$};
\draw (1, 14) node {\tiny $\infty$};
\draw<30-> (1, 14) node {\tiny $13$};
\draw (2, 14) node {\tiny $\infty$};
\draw<30-> (2, 14) node {\tiny $12$};
\draw (3, 14) node {\tiny $\infty$};
\draw<30-> (3, 14) node {\tiny $11$};
\draw (4, 14) node {\tiny $10$};
\draw (5, 14) node {\tiny $9$};
\draw (6, 14) node {\tiny $8$};
\draw (7, 14) node {\tiny $7$};
\draw (8, 14) node {\tiny $6$};
\draw (9, 14) node {\tiny $5$};
\draw (10, 14) node {\tiny $4$};
\draw (11, 14) node {\tiny $3$};
\draw (12, 14) node {\tiny $3$};
\draw (13, 14) node {\tiny $3$};
\draw (14, 14) node {\tiny $4$};
\draw (15, 14) node {\tiny $3$};
\draw (16, 14) node {\tiny $4$};
\draw (17, 14) node {\tiny $3$};
\draw (18, 14) node {\tiny $4$};
\draw (19, 14) node {\tiny $5$};
\draw (20, 14) node {\tiny $6$};
\draw (0, 15) node {\tiny $\infty$};
\draw<30-> (0, 15) node {\tiny $15$};
\draw (1, 15) node {\tiny $\infty$};
\draw<30-> (1, 15) node {\tiny $14$};
\draw (2, 15) node {\tiny $\infty$};
\draw<30-> (2, 15) node {\tiny $13$};
\draw (3, 15) node {\tiny $\infty$};
\draw<30-> (3, 15) node {\tiny $12$};
\draw (4, 15) node {\tiny $\infty$};
\draw<30-> (4, 15) node {\tiny $11$};
\draw (5, 15) node {\tiny $10$};
\draw (6, 15) node {\tiny $9$};
\draw (7, 15) node {\tiny $8$};
\draw (8, 15) node {\tiny $7$};
\draw (9, 15) node {\tiny $6$};
\draw (10, 15) node {\tiny $5$};
\draw (11, 15) node {\tiny $4$};
\draw (12, 15) node {\tiny $4$};
\draw (13, 15) node {\tiny $3$};
\draw (14, 15) node {\tiny $3$};
\draw (15, 15) node {\tiny $2$};
\draw (16, 15) node {\tiny $3$};
\draw (17, 15) node {\tiny $2$};
\draw (18, 15) node {\tiny $3$};
\draw (19, 15) node {\tiny $4$};
\draw (20, 15) node {\tiny $5$};
\draw (0, 16) node {\tiny $\infty$};
\draw<30-> (0, 16) node {\tiny $16$};
\draw (1, 16) node {\tiny $\infty$};
\draw<30-> (1, 16) node {\tiny $15$};
\draw (2, 16) node {\tiny $\infty$};
\draw<30-> (2, 16) node {\tiny $14$};
\draw (3, 16) node {\tiny $\infty$};
\draw<30-> (3, 16) node {\tiny $13$};
\draw (4, 16) node {\tiny $\infty$};
\draw<30-> (4, 16) node {\tiny $12$};
\draw (5, 16) node {\tiny $\infty$};
\draw<30-> (5, 16) node {\tiny $11$};
\draw (6, 16) node {\tiny $10$};
\draw (7, 16) node {\tiny $9$};
\draw (8, 16) node {\tiny $8$};
\draw (9, 16) node {\tiny $7$};
\draw (10, 16) node {\tiny $6$};
\draw (11, 16) node {\tiny $5$};
\draw (12, 16) node {\tiny $4$};
\draw (13, 16) node {\tiny $3$};
\draw (14, 16) node {\tiny $2$};
\draw (15, 16) node {\tiny $1$};
\draw (16, 16) node {\tiny $2$};
\draw (17, 16) node {\tiny $1$};
\draw (18, 16) node {\tiny $2$};
\draw (19, 16) node {\tiny $3$};
\draw (20, 16) node {\tiny $4$};
\draw (0, 17) node {\tiny $\infty$};
\draw<30-> (0, 17) node {\tiny $17$};
\draw (1, 17) node {\tiny $\infty$};
\draw<30-> (1, 17) node {\tiny $16$};
\draw (2, 17) node {\tiny $\infty$};
\draw<30-> (2, 17) node {\tiny $15$};
\draw (3, 17) node {\tiny $\infty$};
\draw<30-> (3, 17) node {\tiny $14$};
\draw (4, 17) node {\tiny $\infty$};
\draw<30-> (4, 17) node {\tiny $13$};
\draw (5, 17) node {\tiny $\infty$};
\draw<30-> (5, 17) node {\tiny $12$};
\draw (6, 17) node {\tiny $\infty$};
\draw<30-> (6, 17) node {\tiny $11$};
\draw (7, 17) node {\tiny $10$};
\draw (8, 17) node {\tiny $9$};
\draw (9, 17) node {\tiny $8$};
\draw (10, 17) node {\tiny $7$};
\draw (11, 17) node {\tiny $6$};
\draw (12, 17) node {\tiny $5$};
\draw (13, 17) node {\tiny $4$};
\draw (14, 17) node {\tiny $3$};
\draw (15, 17) node {\tiny $2$};
\draw (16, 17) node {\tiny $2$};
\draw (17, 17) node {\tiny $1$};
\draw (18, 17) node {\tiny $2$};
\draw (19, 17) node {\tiny $2$};
\draw (20, 17) node {\tiny $3$};
\draw (0, 18) node {\tiny $\infty$};
\draw<30-> (0, 18) node {\tiny $18$};
\draw (1, 18) node {\tiny $\infty$};
\draw<30-> (1, 18) node {\tiny $17$};
\draw (2, 18) node {\tiny $\infty$};
\draw<30-> (2, 18) node {\tiny $16$};
\draw (3, 18) node {\tiny $\infty$};
\draw<30-> (3, 18) node {\tiny $15$};
\draw (4, 18) node {\tiny $\infty$};
\draw<30-> (4, 18) node {\tiny $14$};
\draw (5, 18) node {\tiny $\infty$};
\draw<30-> (5, 18) node {\tiny $13$};
\draw (6, 18) node {\tiny $\infty$};
\draw<30-> (6, 18) node {\tiny $12$};
\draw (7, 18) node {\tiny $\infty$};
\draw<30-> (7, 18) node {\tiny $11$};
\draw (8, 18) node {\tiny $10$};
\draw (9, 18) node {\tiny $9$};
\draw (10, 18) node {\tiny $8$};
\draw (11, 18) node {\tiny $7$};
\draw (12, 18) node {\tiny $6$};
\draw (13, 18) node {\tiny $5$};
\draw (14, 18) node {\tiny $4$};
\draw (15, 18) node {\tiny $3$};
\draw (16, 18) node {\tiny $2$};
\draw (17, 18) node {\tiny $1$};
\draw (18, 18) node {\tiny $2$};
\draw (19, 18) node {\tiny $1$};
\draw (20, 18) node {\tiny $2$};
\draw (0, 19) node {\tiny $\infty$};
\draw<30-> (0, 19) node {\tiny $19$};
\draw (1, 19) node {\tiny $\infty$};
\draw<30-> (1, 19) node {\tiny $18$};
\draw (2, 19) node {\tiny $\infty$};
\draw<30-> (2, 19) node {\tiny $17$};
\draw (3, 19) node {\tiny $\infty$};
\draw<30-> (3, 19) node {\tiny $16$};
\draw (4, 19) node {\tiny $\infty$};
\draw<30-> (4, 19) node {\tiny $15$};
\draw (5, 19) node {\tiny $\infty$};
\draw<30-> (5, 19) node {\tiny $14$};
\draw (6, 19) node {\tiny $\infty$};
\draw<30-> (6, 19) node {\tiny $13$};
\draw (7, 19) node {\tiny $\infty$};
\draw<30-> (7, 19) node {\tiny $12$};
\draw (8, 19) node {\tiny $\infty$};
\draw<30-> (8, 19) node {\tiny $11$};
\draw (9, 19) node {\tiny $10$};
\draw (10, 19) node {\tiny $9$};
\draw (11, 19) node {\tiny $8$};
\draw (12, 19) node {\tiny $7$};
\draw (13, 19) node {\tiny $6$};
\draw (14, 19) node {\tiny $5$};
\draw (15, 19) node {\tiny $4$};
\draw (16, 19) node {\tiny $3$};
\draw (17, 19) node {\tiny $2$};
\draw (18, 19) node {\tiny $1$};
\draw (19, 19) node {\tiny $0$};
\draw (20, 19) node {\tiny $1$};
\draw (0, 20) node {\tiny $\infty$};
\draw<30-> (0, 20) node {\tiny $20$};
\draw (1, 20) node {\tiny $\infty$};
\draw<30-> (1, 20) node {\tiny $19$};
\draw (2, 20) node {\tiny $\infty$};
\draw<30-> (2, 20) node {\tiny $18$};
\draw (3, 20) node {\tiny $\infty$};
\draw<30-> (3, 20) node {\tiny $17$};
\draw (4, 20) node {\tiny $\infty$};
\draw<30-> (4, 20) node {\tiny $16$};
\draw (5, 20) node {\tiny $\infty$};
\draw<30-> (5, 20) node {\tiny $15$};
\draw (6, 20) node {\tiny $\infty$};
\draw<30-> (6, 20) node {\tiny $14$};
\draw (7, 20) node {\tiny $\infty$};
\draw<30-> (7, 20) node {\tiny $13$};
\draw (8, 20) node {\tiny $\infty$};
\draw<30-> (8, 20) node {\tiny $12$};
\draw (9, 20) node {\tiny $\infty$};
\draw<30-> (9, 20) node {\tiny $11$};
\draw (10, 20) node {\tiny $10$};
\draw (11, 20) node {\tiny $9$};
\draw (12, 20) node {\tiny $8$};
\draw (13, 20) node {\tiny $7$};
\draw (14, 20) node {\tiny $6$};
\draw (15, 20) node {\tiny $5$};
\draw (16, 20) node {\tiny $4$};
\draw (17, 20) node {\tiny $3$};
\draw (18, 20) node {\tiny $2$};
\draw (19, 20) node {\tiny $1$};
\draw (20, 20) node {\tiny $0$};

}

        \filldraw<12> (12.5,8.5) circle(0.1);

        \fill<14>[opacity=0.15] (-0.5,-0.5) rectangle (10,8);

        \end{tikzpicture}
    \end{center}
  \end{column}%
\end{columns}%

\end{frame}


\begin{frame}{Algorithms on Distance Matrices}
    \textbf{Construction:}
    \begin{itemize}
        \item  $\Ohtilde(n^2)$ time via multiple-source shortest paths in planar graphs \where{Kle05}.
    \end{itemize}

    \pause

    \bigskip

    \textbf{Storage:} 

    \begin{itemize}
        \item  $\Oh(n^2)$ space for general weights;
        \item  $\Oh(nW)$ space for weights in $\{0,\ldots,W\}$ via core-sparse representation \where{Rus10}.
    \end{itemize}

    \pause

    \bigskip

    \textbf{Stitching:} $D_{X,Y}, D_{X',Y} \leadsto D_{XX',Y}$

    \begin{itemize}
        \item  $\Oh(n^2)$ time for general weights via min-plus multiplication of Monge matrices \where{SMAWK87};
        \item  $\Oh(nW\log n)$ time for weights in $\{0,\ldots,W\}$ using core-sparse multiplication \where{GG\textbf{K}25}.
    \end{itemize}

\end{frame}

\begin{frame}<-3,6->{Hierarchical Box Decomposition}
    \vspace*{-0.32cm}
    \begin{columns}
        \begin{column}{0.5\textwidth}
            \begin{itemize}
                \item<1-> Optimal alignments stay within a band of width $\Oh(k)$.
                \item<2-> Cover the band with $\Oh(\frac{n}{k})$ boxes of size $\Oh(k)\times \Oh(k)$.
                \item<3-> Optimal alignments cross all separators~$V_i$.
                \item<4-> Since $\sed(X) \le k$, then $X$ consists of $\Oh(k)$ parts with period $\Theta(k)$ each.
                \item<7-> Compute distance matrices for all $\Oh(k)$ distinct boxes:
                \begin{itemize}
                    \item<8-> Use weighted grammars to guide a hierarchical box decomposition.
                    \item<9-> One can achieve $\Oh(k^2/\ell)$ distinct boxes of size $\Theta(\ell)\times \Theta(\ell)$.
                    \item<10-> Stitch distance matrices bottom up.
                \end{itemize}
                \item<11-> Finally, compute $\Oh(k\log n)$ products.
            \end{itemize}
        \end{column}
        \begin{column}{0.5\textwidth}
            \begin{center}
            \begin{tikzpicture}[transform canvas={scale=0.55}, y=-1cm, xshift=-5.3cm, yshift=6cm]
                \bigpicture{
                    
\def\shft{0}
\def\W{12}
\def\H{12}
\def\myi{2}
\def\cnt{6}
\def\gp{2}
\def\kp{1.5}
\def\charwidth{1 / 15}
%\draw[black,thin] (-1,-1) grid (\H + 1,\W + 1);

%0/gray,1/red,2/violet,3/blue,4/magenta,5/orange,6/green,7/yellow

\foreach \x/\c/\d/\e in {0/gray/darkblue/gray,\gp/violet/darkblue/darkblue,2*\gp/orange/purple/purple,3*\gp/magenta/purple/purple,4*\gp/yellow/purple/purple,5*\gp/yellow/purple/yellow} {
    \draw<2-4>[fill=\c!40,thick] (\x, {max(\x - \kp, 0)}) rectangle (\x + \gp, {min(\x + \gp + \kp, \H)});
    \draw<5->[fill=\e!40, thick] (\x, {max(\x - \kp, 0)}) rectangle (\x + \gp, {min(\x + \gp + \kp, \H)});
    \draw<8->[densely dotted,step=0.5cm] (\x, {max(\x - \kp, 0)}) grid (\x + \gp, {min(\x + \gp + \kp, \H)});
}

\onslide<9>{
  \foreach \y in {0,.5,...,3} {
      \foreach \x in {0,0.5,...,1.5} {
        \pgfmathrandominteger{\r}{0}{3}
      % map 0..3 to the four colors
        \ifcase\r\relax
        \def\cellcolor{gray}%
        \or
        \def\cellcolor{darkblue}%
        \or
        \def\cellcolor{purple}%
        \or
        \def\cellcolor{yellow}%
        \fi
          \path[fill=\cellcolor!40] (\x,\y) rectangle ++(.5,.5); 
      }
  }
    \foreach \y in {0,0.5,...,4.5} {
      \foreach \x in {0,0.5,...,1.5} {
        \pgfmathrandominteger{\r}{0}{3}
      % map 0..3 to the four colors
        \ifcase\r\relax
        \def\cellcolor{gray}%
        \or
        \def\cellcolor{darkblue}%
        \or
        \def\cellcolor{purple}%
        \or
        \def\cellcolor{yellow}%
        \fi
          \path[fill=\cellcolor!40] (2+\x,.5+\y) rectangle ++(.5,.5); 
      }
  }
      \foreach \y in {0,0.5,...,4.5} {
      \foreach \x in {0,0.5,...,1.5} {
        \pgfmathrandominteger{\r}{0}{3}
      % map 0..3 to the four colors
        \ifcase\r\relax
        \def\cellcolor{gray}%
        \or
        \def\cellcolor{darkblue}%
        \or
        \def\cellcolor{purple}%
        \or
        \def\cellcolor{yellow}%
        \fi
          \path[fill=\cellcolor!40] (4+\x,2.5+\y) rectangle ++(.5,.5); 
          \path[fill=\cellcolor!40] (6+\x,4.5+\y) rectangle ++(.5,.5); 
          \path[fill=\cellcolor!40] (8+\x,6.5+\y) rectangle ++(.5,.5); 

      }
  }

    \foreach \y in {8.5,9,...,11.5} {
      \foreach \x in {10,10.5,...,11.5} {
        \pgfmathrandominteger{\r}{0}{3}
      % map 0..3 to the four colors
        \ifcase\r\relax
        \def\cellcolor{gray}%
        \or
        \def\cellcolor{darkblue}%
        \or
        \def\cellcolor{purple}%
        \or
        \def\cellcolor{yellow}%
        \fi
          \path[fill=\cellcolor!40] (\x,\y) rectangle ++(.5,.5); 
      }
  }
}

%\foreach \x/\c in {} {
%    \draw<2-4>[fill=\c!40, thick] (\x, {max(\x - \kp, 0)}) rectangle (\x + \gp, {min(\x + \gp + \kp, \H)});
%    \draw<5->[fill=darkblue!40, thick] (\x, {max(\x - \kp, 0)}) rectangle (\x + \gp, {min(\x + \gp + \kp, \H)});
%}

%\foreach \x in {4*\gp} {
%    \draw<2-5>[fill=darkblue!40, thick] (\x, {max(\x - \kp, 0)}) rectangle (\x + \gp, {min(\x + \gp + \kp, \H)});
%    \draw<6->[fill=darkblue!70, thick] (\x, {max(\x - \kp, 0)}) rectangle (\x + \gp, {min(\x + \gp + \kp, \H)});
%}

\draw (\kp,0.1) -- (\kp,-0.1) node[above] {$k$};
\draw (0.1,\kp) -- (-0.1,\kp) node[left] {$k$};
\draw[fill=darkgreen,fill opacity=0.1] (0, 0) -- (\kp, 0) -- (\H, \W - \kp) -- (\H, \W) -- (\H - \kp, \W) -- (0, \kp) -- (0, 0);

\onslide<2->{
\foreach \x in {0,\gp,...,{\the\numexpr \W - \gp}}
    \draw (\x, {max(\x - \kp, 0)}) rectangle (\x + \gp, {min(\x + \kp + \gp, \H)});
}

\draw[thick] (0, 0) rectangle (\H, \W);

\onslide<2->{
\draw (0,0.1) -- (0,0) node[above] {$x_0$};
\draw (\W,0.1) -- (\W,0) node[above] {$x_{\cnt}$};
\foreach \x in {1,2,...,{\the\numexpr \cnt - 1}}
    \draw (\x * \gp,0.1) -- (\x * \gp, -0.1) node[above] {$x_{\x}$};


\draw[dashed] (\myi * \gp + \gp, 0) -- (\myi * \gp + \gp, \myi * \gp);
\draw[dashed] (0,\myi * \gp - \kp) node[left] {$x_{\myi} - k$} -- (\myi * \gp,\myi * \gp - \kp);
\draw[dashed] (0,\myi * \gp + \gp + \kp) node[left] {$x_{\the\numexpr \myi + 1} + k$} -- (\myi * \gp,\myi * \gp + \kp+ \gp);
}

\node<3->[left, black] at (\myi * \gp, \myi * \gp) {$V_{\myi}$};
\node<3->[right, black] at (\myi * \gp + \gp, \myi * \gp + \gp) {$V_{\the\numexpr \myi + 1}$};

\draw<3->[very thick, darkgreen] (\myi * \gp, \myi * \gp - \kp) -- (\myi * \gp, \myi * \gp + \kp);
\draw<3->[very thick, darkgreen] (\myi * \gp + \gp, \myi * \gp+\gp-\kp) -- (\myi * \gp + \gp, \myi * \gp + \gp + \kp);

%\node<4->[darkred] at (\myi * \gp + \gp / 2, \myi * \gp + \gp / 2) {\Large $D_{\myi, {\the\numexpr \myi + 1}}$};


%\draw<6->[fill=orange, thick,opacity=0.4] (4.2 * \gp - 1.0 * \gp * \charwidth, 0) rectangle (4.2 * \gp + 1.0 * \gp * \charwidth, \H);
%\draw<7->[fill=orange, thick,opacity=0.4] (0, 2.5 * \gp - 1.0 * \gp * \charwidth) rectangle (\W, 2.5 * \gp + 1.0 * \gp * \charwidth);

                }
            \end{tikzpicture}
            \end{center}
        \end{column}
    \end{columns}
\end{frame}


\againframe<6>{toc}


\begin{frame}{Conclusions}
    \textbf{Contribution:}
    \begin{itemize}
        \item A new scheme for bounded edit distance computation.
        \item Useful in many settings (quantum, dynamic, weighed).
        \item Yields near-optimal algorithms in many parameter regimes.
    \end{itemize}

    \bigskip
    \pause
    \textbf{High-level scheme:}
    \begin{enumerate}
        \item Computing bounded edit distance reduces to instances with self-edit distance $\Oh(k)$.
        \item Instances with small self-edit distance are repetitive.
        \item Stitching distance matrices via min-plus multiplication of structured Monge matrices.
    \end{enumerate}

\end{frame}


\begin{frame}{Open Problems}
    \textbf{Bridge the remaining gaps:}
    \begin{enumerate}
        \item $n^2$ vs $n^{1.5}$ for quantum edit distance.
        \item $n+k^{2.5}$ vs $n+\sqrt{nk^3}$ for weighted edit distance with $\sqrt[3]{n} \le k \le \sqrt{n}$.
        \item $n+k^2$ vs $n+k^2W$ for weighted edit distance with integer weights.
    \end{enumerate}

    \bigskip
    \bigskip
    \bigskip
    \pause
    \textbf{Boundary-to-boundary distance queries in directed grids:}\\
    {\footnotesize \hfill No diagonal edges, arbitrary positive weights on horizontal and vertical edges.}
    \begin{enumerate}
        \item Static algorithm for repetitive grids: is $n^2$ needed for preprocessing plus $n$ queries?
        \item Dynamic algorithm for integer weights $\{1,\ldots,W\}$? 
        \begin{itemize}
            \item Edge cost updates?
            \item Row/column updates?
        \end{itemize}
        \item Dynamic reachability queries ($\{1,\infty\}$ weights)?
    \end{enumerate}

        \begin{tikzpicture}[overlay,remember picture]
      \node<3->[text=MPIgreen] at ([xshift=2.7cm,yshift=.5cm]current page.center){\Huge Thank you!};
      \end{tikzpicture}%

\end{frame}


\end{document}  
